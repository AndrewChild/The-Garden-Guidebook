

\thispagestyle{empty}
\colorlet{shadecolor}{\chapterColor}
\chapter{Garden Cliffs}

\fancyhead{}
\lhead[\textcolor{\chapterColor}{\rule[-2pt]{\textwidth}{15pt}}]{\textcolor{\chapterColor}{\rule[-2pt]{\textwidth}{15pt}}\hspace{-\textwidth}\color{white}\hspace{4pt}\protect\thepage\hspace{1ex}-\hspace{1ex}Garden Cliffs}
\rhead[\textcolor{\chapterColor}{\rule[-2pt]{\textwidth}{15pt}}\hspace{-\textwidth}\color{white}Garden Cliffs \protect\thepage \hspace{4pt}]{\textcolor{\chapterColor}{\rule[-2pt]{\textwidth}{15pt}}}
\fancyhead[RO]{}
\fancyhead[RE]{\color{white}Garden Cliffs\hspace{1ex}-\hspace{1ex}\protect\thepage \hspace{4pt}}


\raggedcolumns
\begin{multicols}{2}
\setbox0=\hbox{\includegraphics[width=0.45\linewidth]{./maps/qr//Garden Cliffs_qr.png}}% Store image in \box0
\needspace{\ht0}% Need at least the height of \box0
\begin{center}
\includegraphics[width=0.45\linewidth]{./maps/qr//Garden Cliffs_qr.png}
\end{center}
\begin{center}
\underline{\textcolor{blue}{\href{http://maps.google.com/maps?q=44.43998124232581,-122.57539325959186}{Navigate to this area}}}
\end{center}


\includegraphics[width=\linewidth]{./maps/plots//Garden Cliffs.png}
\end{multicols}
\begin{multicols}{2}
\phantomsection\label{am:The Garden Cliff Overview}
	\setbox0=\hbox{\begin{overpic}[width=0.8\linewidth]{./maps/area/out/Cliffs_c.png}
	\end{overpic}}
	\needspace{\ht0}
	\begin{center}
	\begin{overpic}[width=0.9\linewidth]{./maps/area/out/Cliffs_c.png}
	\end{overpic}
	\end{center}


Located about 3.7 miles down Quatzville Road from Highway 20, park in a small pull out on the river side of the road or park as for the Garden Main and walk a few hundred yards down the road. The approach trail is located directly across from the parking pull out and can be identified by a stone stairway climbing out of the roadside ditch. Follow the meandering trail a few hundered yards to the Garden Cliff.

The Garden Cliffs are the premier sport climbing destination of the Sweethome area. Although the extensive cliff face is visible from the road most of the routes at the cliffs were developed much later than the boulders.

Take note that the top of the cliff is not easily accesible and none of the routes in this area can be top roped without leading. The majority of the routes in this area feature high first bolts which were placed with stick clipping in mind. Don't fret if you forgot your stick clip at home, a community stick lives at the cliffs and can ususally be found near the entrance to the Garden Cliff.\\



\null\newpage

\section{A - The Garden Cliff}\phantomsection\label{sa:The Garden Cliff}

The largest of the area's cliffs is also conveniently the closest to the road. This cliff features varied climbing at grades rangeing from 5.9 to 5.13- in a variety of stlyes. Many sections of the wall are shaded by large maple trees, which provide some relief in the hotter months. See area description for appoach\\




\needspace{10em}
\subsection*{Garden Cliff Right Side}\phantomsection\label{bf:Garden Cliff Right Side}




\needspace{2em}
\phantomsection\label{rt:Bonsai}
\colorbox{Goldenrod!20}{
\parbox{0.95\linewidth}{
\hspace{-1ex}\textbf{$\Box$
1 Bonsai 5.12a \ding{72}\ding{72} 
}}}
\begin{adjustwidth}{1.3em}{}			

20', Sport, 2 bolts. Climb a techy dihedral just to the right of a low roof. This could almost be bouldered.
  (No Topo)
\end{adjustwidth}




\needspace{2em}
\phantomsection\label{rt:Ladybug}
\colorbox{RoyalBlue!20}{
\parbox{0.95\linewidth}{
\hspace{-1ex}\textbf{$\Box$
2 Ladybug 5.10d \ding{72} 
}}}
\begin{adjustwidth}{1.3em}{}			

25', Sport, 5 bolts. This bewilderingly difficult climb follows the shallow dihedral just to the right of a tree.
  (No Topo)
\end{adjustwidth}



	\end{multicols}
\phantomsection\label{tp:Garden Cliff Right Side}
	\setbox0=\hbox{\begin{overpic}[width=0.8\linewidth]{./maps/topos//cliffs/cliffs_main_c.png}
	\end{overpic}}
	\needspace{\ht0}
	\begin{center}
	\begin{overpic}[width=0.9\linewidth]{./maps/topos//cliffs/cliffs_main_c.png}
	\end{overpic}
	\end{center}

	\begin{multicols}{2}

\needspace{2em}
\phantomsection\label{rt:John Henry's Hammer}
\colorbox{RoyalBlue!20}{
\parbox{0.95\linewidth}{
\hspace{-1ex}\textbf{$\Box$
3 John Henry's Hammer 5.10c/d \ding{72}\ding{72} 
}}}
\begin{adjustwidth}{1.3em}{}			

50', Sport, 6 bolts. Start on a tombstone flake and follow a crack system right before meandering back right and finishing in a short dihedral. This route was originally climbed on gear.
\end{adjustwidth}


\begin{adjustwidth}{0.5cm}{}				
\needspace{4em}
\textbf{Variations:} \newline

\needspace{2em}
\phantomsection\label{vr:John to Snug Linkup}
\colorbox{RoyalBlue!20}{
\parbox{0.95\linewidth}{
\hspace{-1ex}\textbf{$\Box$
3a John to Snug Linkup 5.10c/d \ding{72}\ding{72} 
}}}
\begin{adjustwidth}{1.3em}{}			

50', Sport, 5 bolts. Start on John Henry's Hammer and trend diagonal left throught the broken section of rock linking into Snug as a Snail. Links the easiest sections of every route to the top of the wall.
  (No Topo)
\end{adjustwidth}



\end{adjustwidth}


\needspace{2em}
\phantomsection\label{rt:Yggdrasil}
\colorbox{RoyalBlue!20}{
\parbox{0.95\linewidth}{
\hspace{-1ex}\textbf{$\Box$
4 Yggdrasil 5.11a \ding{72}\ding{72}\ding{72} 
}}}
\begin{adjustwidth}{1.3em}{}			

50', Sport, 6 bolts. Start as for John Henry's Hammer, but stay left after the first bolt.
\end{adjustwidth}




\needspace{2em}
\phantomsection\label{rt:Scorpion Revenge}
\colorbox{RoyalBlue!20}{
\parbox{0.95\linewidth}{
\hspace{-1ex}\textbf{$\Box$
5 Scorpion Revenge 5.11b \ding{72}\ding{72} 
}}}
\begin{adjustwidth}{1.3em}{}			

50', Sport, 6 bolts. Starts with a few bouldery moves up a left facing ramp then continues through small crimps before some bigger moves on jugs. Many of the crimps on this route have broken and are much smaller than they used to be.
\end{adjustwidth}




\needspace{2em}
\phantomsection\label{rt:Snug as a Snail}
\colorbox{RoyalBlue!20}{
\parbox{0.95\linewidth}{
\hspace{-1ex}\textbf{$\Box$
6 Snug as a Snail 5.11c \ding{72}\ding{72} 
}}}
\begin{adjustwidth}{1.3em}{}			

50', Sport, 5 bolts. Climbs an obvious flared dihedral before gaining good holds higher up.
\end{adjustwidth}




\needspace{2em}
\phantomsection\label{rt:Scorpion Hitchhikers Toilet Bowl Odyssey}
\colorbox{RoyalBlue!20}{
\parbox{0.95\linewidth}{
\hspace{-1ex}\textbf{$\Box$
7 Scorpion Hitchhikers Toilet Bowl Odyssey 5.11b \ding{72}\ding{72}\ding{72} 
}}}
\begin{adjustwidth}{1.3em}{}			

50', Sport, 5 bolts. Starting just left of Snug pull a few jugy moves to gain a left leaning crescent and follow it to an exhillerating dynamic move after the third bolt. After a jug rest continue through another 15' of pumpy climbing until you gain a no hands rest on a ledge at the top of the wall. Why the anchor is not accessible from this ledge is a mystery a bonus few techy moves leads to a tenuous stance at the anchor.
\end{adjustwidth}




\needspace{2em}
\phantomsection\label{rt:Daring to Fly}
\colorbox{RoyalBlue!20}{
\parbox{0.95\linewidth}{
\hspace{-1ex}\textbf{$\Box$
8 Daring to Fly 5.11d \ding{72}\ding{72}\ding{72} 
}}}
\begin{adjustwidth}{1.3em}{}			

55', Sport, 7 bolts. Start on the left side of a small cave and climb the aesthtic pillar.
\end{adjustwidth}




\needspace{2em}
\phantomsection\label{rt:Community}
\colorbox{green!20}{
\parbox{0.95\linewidth}{
\hspace{-1ex}\textbf{$\Box$
9 Community 5.9 \ding{72} 
}}}
\begin{adjustwidth}{1.3em}{}			

55', Sport, 7 bolts. Starting in the same alcove as Daring to Fly climb the right facing ramp up and left to a ledgy top. This route has a reputation for being weird and techy not the easiest lead at the grade. There are also several cracks where you could practice gear placements on route.
\end{adjustwidth}




\needspace{2em}
\phantomsection\label{rt:Blackberry Jam}
\colorbox{RoyalBlue!20}{
\parbox{0.95\linewidth}{
\hspace{-1ex}\textbf{$\Box$
10 Blackberry Jam 5.10-*  
}}}
\begin{adjustwidth}{1.3em}{}			

45', Trad, gear to 3". Climb a dirty right facing dihedral and link into a less pleasant fist crack up and right. Finishes at a bolted anchor.
  (No Topo)
\end{adjustwidth}




\needspace{2em}
\phantomsection\label{rt:Anaphylactic Shock}
\colorbox{Goldenrod!20}{
\parbox{0.95\linewidth}{
\hspace{-1ex}\textbf{$\Box$
11 Anaphylactic Shock 5.12a \ding{72} 
}}}
\begin{adjustwidth}{1.3em}{}			

40', Mixed, 3 bolts and gear to 0.75". Climb a left leaning crack to an easy mantle at the top of a small roof. Enjoy a no hands rest before a difficult boulder problem at the anchor.
\end{adjustwidth}




\needspace{2em}
\phantomsection\label{rt:Fight Club (Round Two)}
\colorbox{Goldenrod!20}{
\parbox{0.95\linewidth}{
\hspace{-1ex}\textbf{$\Box$
12 Fight Club (Round Two) 5.12b \ding{72}\ding{72}\ding{72} 
}}}
\begin{adjustwidth}{1.3em}{}			

50', Sport, 7 bolts. Not to be confused with Fight Club (the boulder problem) or Fight Club 2 (the boulder problem), Fight Club Round Two is one of the primeir sport climbing test pieces at the cliffs. Starts on a right facing corner before traversing under the roof until you can grapple your way up to the techy headwall. The crux section of the route is equipped with perma draws, get on it!
\end{adjustwidth}




\needspace{2em}
\phantomsection\label{rt:Cutting Crack}
\colorbox{green!20}{
\parbox{0.95\linewidth}{
\hspace{-1ex}\textbf{$\Box$
13 Cutting Crack 5.9 \ding{72} 
}}}
\begin{adjustwidth}{1.3em}{}			

20', Trad, gear to 2". Follow a short hand crack until you can clip one of the perma draws for Fight Club. Lower here or continue up.
\end{adjustwidth}




\needspace{2em}
\phantomsection\label{rt:Butterfly Effect}
\colorbox{Goldenrod!20}{
\parbox{0.95\linewidth}{
\hspace{-1ex}\textbf{$\Box$
14 Butterfly Effect 5.13a/b  
}}}
\begin{adjustwidth}{1.3em}{}			

40', Sport, 6 bolts. Climbs more or less straight up through a low bolcky ledge followed by thin crimps and a bouldery roof pull. Reportedly climbs like low 5.12 endurance into V7/8 with no rest. The middle of the route is equiped with permas.
\end{adjustwidth}




\needspace{2em}
\phantomsection\label{rt:Slithering Skink}
\colorbox{RoyalBlue!20}{
\parbox{0.95\linewidth}{
\hspace{-1ex}\textbf{$\Box$
15 Slithering Skink 5.10d \ding{72}\ding{72}\ding{72} 
}}}
\begin{adjustwidth}{1.3em}{}			

40', Sport, 6 bolts. Start as for butterfly effect but cut left at the blocky ledge and traverse into a big corner. Follow good holds up and overhang and into a techy sequence through a short dihedral.
\end{adjustwidth}



	\end{multicols}
\phantomsection\label{tp:Garden Cliff Skink Area}
	\setbox0=\hbox{\begin{overpic}[width=0.8\linewidth]{./maps/topos//cliffs/skink_c.png}
	\end{overpic}}
	\needspace{\ht0}
	\begin{center}
	\begin{overpic}[width=0.9\linewidth]{./maps/topos//cliffs/skink_c.png}
	\end{overpic}
	\end{center}

	\begin{multicols}{2}

\needspace{2em}
\phantomsection\label{rt:Stasis Chamber}
\colorbox{Goldenrod!20}{
\parbox{0.95\linewidth}{
\hspace{-1ex}\textbf{$\Box$
16 Stasis Chamber 5.12b \ding{72}\ding{72} 
}}}
\begin{adjustwidth}{1.3em}{}			

40', Sport, 6 bolts. Climb a steep prow to the left of the slithering skink corner. After gaining the big ledge rejoin with skink.
\end{adjustwidth}


\begin{adjustwidth}{0.5cm}{}				
\needspace{4em}
\textbf{Variations:} \newline

\needspace{2em}
\phantomsection\label{vr:Lazarus}
\colorbox{Goldenrod!20}{
\parbox{0.95\linewidth}{
\hspace{-1ex}\textbf{$\Box$
16a Lazarus 5.12c \ding{72}\ding{72} 
}}}
\begin{adjustwidth}{1.3em}{}			

40', Sport, 6 bolts. Climb Stasis to the ledge then instead of rolling onto the ledge traverse left around the corner and link into the finish of Wildlings.
\end{adjustwidth}



\end{adjustwidth}


\needspace{2em}
\phantomsection\label{rt:Wildlings}
\colorbox{RoyalBlue!20}{
\parbox{0.95\linewidth}{
\hspace{-1ex}\textbf{$\Box$
17 Wildlings 5.11d \ding{72}\ding{72} 
}}}
\begin{adjustwidth}{1.3em}{}			

40', Sport, 6 bolts. Traverse left under the Stasis chamber prow into a sustained dihedral.
\end{adjustwidth}




\needspace{2em}
\phantomsection\label{rt:Rain Shadow}
\colorbox{RoyalBlue!20}{
\parbox{0.95\linewidth}{
\hspace{-1ex}\textbf{$\Box$
18 Rain Shadow 5.11a/b \ding{72} 
}}}
\begin{adjustwidth}{1.3em}{}			

30', Sport, 3 bolts. Pull a few juggy moves through broken rock down low and negotiate a techy dihedral to clip the chains.
\end{adjustwidth}




\needspace{2em}
\phantomsection\label{rt:Lenticular Cloud Project}
\colorbox{black!20}{
\parbox{0.95\linewidth}{
\hspace{-1ex}\textbf{$\Box$
19 Lenticular Cloud Project 5.?  
}}}
\begin{adjustwidth}{1.3em}{}			

Open Project. 40', Sport, 8 bolts. Start on Rain shadow but traverse left after the second bolt and follow a weakness out the big roof, long slings are required on several bolts to prevent rope drag. A blank section immediately after the roof has foiled all ascent attempts so far. This project has been opend by it's developer with the caveat that he requests the FA to name the route "Lenticular Cloud".
\end{adjustwidth}




\needspace{2em}
\phantomsection\label{rt:Vine Project}
\colorbox{black!20}{
\parbox{0.95\linewidth}{
\hspace{-1ex}\textbf{$\Box$
20 Vine Project 5.?  
}}}
\begin{adjustwidth}{1.3em}{}			

60', Sport, 9 bolts. Open Project. Starts on the far end of the rain shadow ledge. This route was bolted and climbed as a dry tooling route, maybe it goes on fingers as well?
\end{adjustwidth}




	\end{multicols}
\phantomsection\label{tp:Garden Cliff Castle Black Area}
  \begin{landscape}
	\includepdf[angle=90, picturecommand*={}]{./maps/topos//cliffs/castle2_c.pdf}
  \end{landscape}

	\begin{multicols}{2}



\needspace{10em}
\subsection*{Garden Cliff Middle}\phantomsection\label{bf:Garden Cliff Middle}




\needspace{2em}
\phantomsection\label{rt:Heirloom}
\colorbox{Goldenrod!20}{
\parbox{0.95\linewidth}{
\hspace{-1ex}\textbf{$\Box$
21 Heirloom 5.13c \ding{72}\ding{72} 
}}}
\begin{adjustwidth}{1.3em}{}			

70', Sport, ? bolts. Climbs an aesthic black arête.
\end{adjustwidth}


\begin{adjustwidth}{0.5cm}{}				
\needspace{4em}
\textbf{Variations:} \newline

\needspace{2em}
\phantomsection\label{vr:Heirloom Left Project}
\colorbox{black!20}{
\parbox{0.95\linewidth}{
\hspace{-1ex}\textbf{$\Box$
21a Heirloom Left Project 5.?  
}}}
\begin{adjustwidth}{1.3em}{}			

The original line of the Heirloom project stays left when the Heirloom boltline veers right at the top.
\end{adjustwidth}



\end{adjustwidth}


\needspace{2em}
\phantomsection\label{rt:Chimeras}
\colorbox{Goldenrod!20}{
\parbox{0.95\linewidth}{
\hspace{-1ex}\textbf{$\Box$
22 Chimeras 5.13a  
}}}
\begin{adjustwidth}{1.3em}{}			

70', Sport, 9 bolts. CLimb through the middle of a big scoop with a bouldery exit. Ignore the first bolt to prevent rope drag.
\end{adjustwidth}




\needspace{2em}
\phantomsection\label{rt:Castle Black}
\colorbox{RoyalBlue!20}{
\parbox{0.95\linewidth}{
\hspace{-1ex}\textbf{$\Box$
23 Castle Black 5.11a \ding{72} 
}}}
\begin{adjustwidth}{1.3em}{}			

50', Sport, ? bolts. Originally this was a somewhat bold trad climb, it has since been bolted. Climb the lower cliff band to a right facing corner with a big ledge half way up. This route is baisically a waterfall in the winter and typically doesn't dry out until mid summer.
\end{adjustwidth}




\needspace{2em}
\phantomsection\label{rt:Littlest Birds}
\colorbox{RoyalBlue!20}{
\parbox{0.95\linewidth}{
\hspace{-1ex}\textbf{$\Box$
24 Littlest Birds 5.11b \ding{72}\ding{72} 
}}}
\begin{adjustwidth}{1.3em}{}			

70', Sport, 9 bolts. Start on Castle Black and cut right after the midway ledge. A techy sequence leads to sustained clibing up a well featured pillar.
\end{adjustwidth}




\needspace{2em}
\phantomsection\label{rt:Seraphim Nachash}
\colorbox{RoyalBlue!20}{
\parbox{0.95\linewidth}{
\hspace{-1ex}\textbf{$\Box$
25 Seraphim Nachash 5.11b/c \ding{72}\ding{72}\ding{72} 
}}}
\begin{adjustwidth}{1.3em}{}			

70', Sport, 10 bolts. Easy moves lead to a no hands rest on a ledge at the top of the lower cliffband. From here pull a crux seqence climbing into a corner followed by a long section of power endurance on good holds.
\end{adjustwidth}




\needspace{2em}
\phantomsection\label{rt:My Empire of Dirt}
\colorbox{Goldenrod!20}{
\parbox{0.95\linewidth}{
\hspace{-1ex}\textbf{$\Box$
26 My Empire of Dirt 5.12b \ding{72}\ding{72}\ding{72} 
}}}
\begin{adjustwidth}{1.3em}{}			

70', Sport, 11 bolts. Climb Seraphim Nachash through its crux then cut left through a physical section of sidepulls and underclings. Continue up a sustained overhaing wall with big pulls between generally decent holds. This is a real power endurance testpiece.
\end{adjustwidth}




\needspace{2em}
\phantomsection\label{rt:Honeycomb Project}
\colorbox{black!20}{
\parbox{0.95\linewidth}{
\hspace{-1ex}\textbf{$\Box$
27 Honeycomb Project 5.?  
}}}
\begin{adjustwidth}{1.3em}{}			

50', Sport, 9 bolts. Open Project. Start on a narrow ledge to the right of the top of the stairs. The section down low has so far never been climbed.
\end{adjustwidth}


\begin{adjustwidth}{0.5cm}{}				
\needspace{4em}
\textbf{Variations:} \newline

\needspace{2em}
\phantomsection\label{vr:Honeycomb Traverse}
\colorbox{Goldenrod!20}{
\parbox{0.95\linewidth}{
\hspace{-1ex}\textbf{$\Box$
27a Honeycomb Traverse 5.12a \ding{72}\ding{72} 
}}}
\begin{adjustwidth}{1.3em}{}			

50', Sport, 9 bolts. Start on Criss Cross Apple Sauce and traverse into Honeycomb after the third bolt. Avoids the blank section down low.
\end{adjustwidth}



\end{adjustwidth}


\needspace{2em}
\phantomsection\label{rt:Criss Cross Applesauce}
\colorbox{RoyalBlue!20}{
\parbox{0.95\linewidth}{
\hspace{-1ex}\textbf{$\Box$
28 Criss Cross Applesauce 5.11c \ding{72}\ding{72} 
}}}
\begin{adjustwidth}{1.3em}{}			

45', Sport, 8 bolts. Start at the top of the stairs. After clipping the third bolt follow a jug rail up and right to hard to decipher crux at the end of a pumpy sequence. Climbing eases substantially after the crux.
\end{adjustwidth}




\needspace{2em}
\phantomsection\label{rt:Vandals in the Graveyard}
\colorbox{Goldenrod!20}{
\parbox{0.95\linewidth}{
\hspace{-1ex}\textbf{$\Box$
29 Vandals in the Graveyard 5.12a \ding{72} 
}}}
\begin{adjustwidth}{1.3em}{}			

45', Sport, 5 bolts. Start on Criss Cross Applesauce but continue straight up after the 3rd bolt. After a short bouldery sequence gain a left facing ramp and follow easy terrain back to the chains of Criss Cross Apple Sauce.
\end{adjustwidth}




	\end{multicols}
\phantomsection\label{tp:Garden Cliff Nest Area}
	\setbox0=\hbox{\begin{overpic}[width=0.8\linewidth]{./maps/topos//cliffs/nest2_c.png}
	\end{overpic}}
	\needspace{\ht0}
	\begin{center}
	\begin{overpic}[width=0.9\linewidth]{./maps/topos//cliffs/nest2_c.png}
	\end{overpic}
	\end{center}

	\begin{multicols}{2}

\needspace{10em}
\subsection*{Garden Cliff Left Side}\phantomsection\label{bf:Garden Cliff Left Side}




\needspace{2em}
\phantomsection\label{rt:Ovulation Send-sation}
\colorbox{Goldenrod!20}{
\parbox{0.95\linewidth}{
\hspace{-1ex}\textbf{$\Box$
30 Ovulation Send-sation 5.12a \ding{72}\ding{72} 
}}}
\begin{adjustwidth}{1.3em}{}			

45', Sport, 6 bolts. Technical climbing leads small holds and pockets off of a low ledge. Joins Fertile crescent after the 4th bolt before a tricky roof pull to gain the anchor.
\end{adjustwidth}


\begin{adjustwidth}{0.5cm}{}				
\needspace{4em}
\textbf{Variations:} \newline

\needspace{2em}
\phantomsection\label{vr:Ovulation Send-sation Extension}
\colorbox{Goldenrod!20}{
\parbox{0.95\linewidth}{
\hspace{-1ex}\textbf{$\Box$
30a Ovulation Send-sation Extension 5.12a*  
}}}
\begin{adjustwidth}{1.3em}{}			

A two bolt extension takes this climb or its neighbor from a ledge to the top of the cliff. Probably doesn't change the grade.
\end{adjustwidth}



\end{adjustwidth}


\needspace{2em}
\phantomsection\label{rt:Fertile Crescent}
\colorbox{Goldenrod!20}{
\parbox{0.95\linewidth}{
\hspace{-1ex}\textbf{$\Box$
31 Fertile Crescent 5.12a \ding{72}\ding{72}\ding{72} 
}}}
\begin{adjustwidth}{1.3em}{}			

45', Sport, 6 bolts. Climb the large left facing crescent feature until you join with Ovulation after the fourth bolt.
\end{adjustwidth}




\needspace{2em}
\phantomsection\label{rt:My Secret Garden}
\colorbox{RoyalBlue!20}{
\parbox{0.95\linewidth}{
\hspace{-1ex}\textbf{$\Box$
32 My Secret Garden 5.11a \ding{72}\ding{72}\ding{72} 
}}}
\begin{adjustwidth}{1.3em}{}			

45', Sport, 6 bolts. Start on a big sloping rail and pull a few moves to reach good edges and manuver your way under a hanging block. Crank some big moves to get up and around the block onto easier terrain.
\end{adjustwidth}




\needspace{2em}
\phantomsection\label{rt:Nest}
\colorbox{RoyalBlue!20}{
\parbox{0.95\linewidth}{
\hspace{-1ex}\textbf{$\Box$
33 Nest 5.10c \ding{72}\ding{72} 
}}}
\begin{adjustwidth}{1.3em}{}			

45', Sport, 6 bolts. Start in a little corner just left of My Secret Garden, Climb more or less straight up. Technical.
\end{adjustwidth}




\needspace{2em}
\phantomsection\label{rt:A Garden Called Peace}
\colorbox{RoyalBlue!20}{
\parbox{0.95\linewidth}{
\hspace{-1ex}\textbf{$\Box$
34 A Garden Called Peace 5.10a \ding{72}\ding{72}\ding{72} 
}}}
\begin{adjustwidth}{1.3em}{}			

45', Sport, 6 bolts. Layback up a huge flake then find a good rest before pulling the crux at a little roof bulge. Take caution multiple people have sprained their ankle falling after the 3rd bolt, a hard catch from an attentive belayer will keep you from hitting the ledge below.
\end{adjustwidth}




\needspace{2em}
\phantomsection\label{rt:Hive}
\colorbox{RoyalBlue!20}{
\parbox{0.95\linewidth}{
\hspace{-1ex}\textbf{$\Box$
35 Hive 5.10c \ding{72} 
}}}
\begin{adjustwidth}{1.3em}{}			

50', Sport, 9 bolts. Hard moves down low are followed by a good rest and a meandering path which climbs both sides of a leaning tower.
\end{adjustwidth}




	\end{multicols}


	\begin{multicols}{2}

\null\newpage

\section{B - Fairy Tale Wall}\phantomsection\label{sa:Fairy Tale Wall}

Following the main trail past the Garden Cliff brings you to a second, much smaller cliff.\\




\needspace{10em}


\phantomsection\label{tp:Fairy Tale Wall}
	\setbox0=\hbox{\begin{overpic}[width=0.8\linewidth]{./maps/topos//cliffs/fairytale2_c.png}
	\end{overpic}}
	\needspace{\ht0}
	\begin{center}
	\begin{overpic}[width=0.9\linewidth]{./maps/topos//cliffs/fairytale2_c.png}
	\end{overpic}
	\end{center}



\needspace{2em}
\phantomsection\label{rt:Baba Yaga}
\colorbox{green!20}{
\parbox{0.95\linewidth}{
\hspace{-1ex}\textbf{$\Box$
1 Baba Yaga 5.9 \ding{72} 
}}}
\begin{adjustwidth}{1.3em}{}			

20', Sport, 4 bolts. Climbing eases up after a few hard moves down low.
\end{adjustwidth}




\needspace{2em}
\phantomsection\label{rt:Death of Koschei the Deathless}
\colorbox{RoyalBlue!20}{
\parbox{0.95\linewidth}{
\hspace{-1ex}\textbf{$\Box$
2 Death of Koschei the Deathless 5.11a \ding{72} 
}}}
\begin{adjustwidth}{1.3em}{}			

20', Sport, 4 bolts. Easy climbing surrounds a one move crux deadpoint.
\end{adjustwidth}




\needspace{2em}
\phantomsection\label{rt:Feather of the Finest Falcon}
\colorbox{green!20}{
\parbox{0.95\linewidth}{
\hspace{-1ex}\textbf{$\Box$
3 Feather of the Finest Falcon 5.8 \ding{72} 
}}}
\begin{adjustwidth}{1.3em}{}			

25', Sport, 4 bolts. A series of blocky ledges leads to a short but sweet wall.
\end{adjustwidth}




\needspace{2em}
\phantomsection\label{rt:Fee-Fi-Fo-Fum}
\colorbox{RoyalBlue!20}{
\parbox{0.95\linewidth}{
\hspace{-1ex}\textbf{$\Box$
4 Fee-Fi-Fo-Fum 5.10c \ding{72}\ding{72} 
}}}
\begin{adjustwidth}{1.3em}{}			

25', Sport, 5 bolts. Follow a technical slab to a cruxy pull on a bulgy protrusion.
\end{adjustwidth}





\null\newpage

\section{C - Cabbage Patch/ Thunderdome}\phantomsection\label{sa:Cabbage Patch/ Thunderdome}

Even futher down the main trail is a third cliff which hosts a good selection of more moderate climbs.\\
\textbf{NOTE: This area is incomplete. Look forward to more information in future revisions of this book or contribute your own knowledge on github.}\\




\needspace{10em}
\subsection*{Cabbage Patch}\phantomsection\label{bf:Cabbage Patch}




\needspace{2em}
\phantomsection\label{rt:Don't Forget the Nooch}
\colorbox{green!20}{
\parbox{0.95\linewidth}{
\hspace{-1ex}\textbf{$\Box$
1 Don't Forget the Nooch 5.4*  
}}}
\begin{adjustwidth}{1.3em}{}			

PLACEHOLDER
  (No Topo)
\end{adjustwidth}




\needspace{2em}
\phantomsection\label{rt:Tabouli}
\colorbox{green!20}{
\parbox{0.95\linewidth}{
\hspace{-1ex}\textbf{$\Box$
2 Tabouli 5.7*  
}}}
\begin{adjustwidth}{1.3em}{}			

PLACEHOLDER
  (No Topo)
\end{adjustwidth}




\needspace{2em}
\phantomsection\label{rt:Babaganoush}
\colorbox{RoyalBlue!20}{
\parbox{0.95\linewidth}{
\hspace{-1ex}\textbf{$\Box$
3 Babaganoush 5.10b*  
}}}
\begin{adjustwidth}{1.3em}{}			

PLACEHOLDER
  (No Topo)
\end{adjustwidth}




\needspace{2em}
\phantomsection\label{rt:Kim Chi Corner}
\colorbox{RoyalBlue!20}{
\parbox{0.95\linewidth}{
\hspace{-1ex}\textbf{$\Box$
4 Kim Chi Corner 5.11a*  
}}}
\begin{adjustwidth}{1.3em}{}			

PLACEHOLDER
  (No Topo)
\end{adjustwidth}




	\end{multicols}
\phantomsection\label{tp:Thunderdome}
	\setbox0=\hbox{\begin{overpic}[width=0.8\linewidth]{./maps/topos//cliffs/thunderDome2_c.png}
	\end{overpic}}
	\needspace{\ht0}
	\begin{center}
	\begin{overpic}[width=0.9\linewidth]{./maps/topos//cliffs/thunderDome2_c.png}
	\end{overpic}
	\end{center}

	\begin{multicols}{2}

\needspace{10em}
\subsection*{Johny Cash's Thunderdome}\phantomsection\label{bf:Johny Cash's Thunderdome}




\needspace{2em}
\phantomsection\label{rt:Where Do You Go When the Sun Goes Down}
\colorbox{RoyalBlue!20}{
\parbox{0.95\linewidth}{
\hspace{-1ex}\textbf{$\Box$
5 Where Do You Go When the Sun Goes Down 5.11a*  
}}}
\begin{adjustwidth}{1.3em}{}			

45', Sport, 8 bolts. Follow a shallow right facing corner to a ledge, then trend left and up over a large detatched flake.
\end{adjustwidth}


\begin{adjustwidth}{0.5cm}{}				
\needspace{4em}
\textbf{Variations:} \newline

\needspace{2em}
\phantomsection\label{vr:A Victem of the Times}
\colorbox{RoyalBlue!20}{
\parbox{0.95\linewidth}{
\hspace{-1ex}\textbf{$\Box$
5a A Victem of the Times 5.10c*  
}}}
\begin{adjustwidth}{1.3em}{}			

45', Sport, 8 bolts. Start on Where Do You Go When the Sun Goes Down and link into the top of I dropped a Man in Reno.
  (No Topo)
\end{adjustwidth}



\end{adjustwidth}


\needspace{2em}
\phantomsection\label{rt:I Dropped a Man in Reno}
\colorbox{green!20}{
\parbox{0.95\linewidth}{
\hspace{-1ex}\textbf{$\Box$
6 I Dropped a Man in Reno 5.7*  
}}}
\begin{adjustwidth}{1.3em}{}			

45', Sport, 11 bolts. Start on the right side of a large flake. Traverse right along a shallow ledge (crossing over Where Do You Go When the Sun Goes Down) to a grove and follow it to the top.
\end{adjustwidth}




\needspace{2em}
\phantomsection\label{rt:I Fell for You Like a Child}
\colorbox{green!20}{
\parbox{0.95\linewidth}{
\hspace{-1ex}\textbf{$\Box$
7 I Fell for You Like a Child 5.8 \ding{72}\ding{72} 
}}}
\begin{adjustwidth}{1.3em}{}			

35', Sport, 6 bolts. Start on the right side of a large flake. From a stance at the top of the flake trend slightly right, off of the flake before continuing up through small pods.
\end{adjustwidth}




\needspace{2em}
\phantomsection\label{rt:I Will Let You Down}
\colorbox{green!20}{
\parbox{0.95\linewidth}{
\hspace{-1ex}\textbf{$\Box$
8 I Will Let You Down 5.6 \ding{72}\ding{72} 
}}}
\begin{adjustwidth}{1.3em}{}			

35', Sport, 6 bolts. Climb the center of the large flake then continue up the right face of a low angle corner.
\end{adjustwidth}




\needspace{2em}
\phantomsection\label{rt:A Million Dollars of Good}
\colorbox{green!20}{
\parbox{0.95\linewidth}{
\hspace{-1ex}\textbf{$\Box$
9 A Million Dollars of Good 5.8 \ding{72}\ding{72} 
}}}
\begin{adjustwidth}{1.3em}{}			

35', Sport, 7/6 bolts. Climb the left corner of the large flake before pulling onto a technical slab. Alternatively you can start as for I Will Let You Down and clip a connnector bolt as you walk across the top of the flake skipping the first two bolts. This route would get full stars if the rock quality were better.
\end{adjustwidth}




\needspace{2em}
\phantomsection\label{rt:Stop Once to Wipe the Sweat Away}
\colorbox{RoyalBlue!20}{
\parbox{0.95\linewidth}{
\hspace{-1ex}\textbf{$\Box$
10 Stop Once to Wipe the Sweat Away 5.10a \ding{72} 
}}}
\begin{adjustwidth}{1.3em}{}			

30', Sport, 5 bolts. Start to the left of the large flake. A short sequence of difficult vertical climbing guards an enjoyable slab. This route is currently severly overgrown.
\end{adjustwidth}






\end{multicols}
\clearpage