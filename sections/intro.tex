\colorlet{shadecolor}{\chapterColor}
\chapter{Introduction}
\markboth{\color{white}Introduction \protect\thepage \hspace{4pt}}{}
\lhead{\textcolor{\chapterColor}{\rule[-2pt]{\textwidth}{15pt}}}
\section{Ameneties}
\subsection*{Toilets}
Barring emergencies digging catholes should be avoided when climbing at the Garden Main area. As a alternative consider driving 1.9 miles back towards sweethome to use the pit toilets outside of sunnyside park. The three minute drive will take roughly the same amount of time as fiding a seculded spot and digging a hole and you won't have to worry about squatting on a patch of poison oak. Likewise all of the areas in this book exist within a 5 minutes drive of a toilet or pit toilet. 
\subsection*{Camping}
Paid campsites can be reserved at Sunny Side Park 1.9 miles away from the Garden Main area. Dispersed camping is allowed on any of the pullouts on Quartzville Creek road East of Green Peter Resivour. Camping is not allowed at the Garden Main or Upper Garden areas.
\section{Local Ethics}
The Garden Main, Armageddon, and Pink Tag areas are located on private land owned by the Cascade Timber Company. The company allows walk in access to their land, but there is no official relationship between the landowners and climbers. The established ethic for climbing on timber land in Oregon is that the owners prefer not to get involved and that climbers should do their best to keep a low enough profile that the land owners don't need to get involved. There are a few specific activies which could threaten access for everyone:\\
\begin{itemize}
\item Building fires or causing fire hazards.\\
\item Parking on or blocking gated forest roads.\\
\item Overnight camping at the climbing areas.\\
\item Failing to obay posted fire closures.\\
\end{itemize}
\subsection*{We like the moss}
The lush moss coverings that adorn the boulders are an essential part of the area's charm. When cleaning boulders try to take a conservative approach and avoid demossing uneccissary parts of the boulder.
\section{History}
PLACEHOLDER
\section{Poison Oak}
The Pink Tag and Armageddon areas are both plagued by poison oak tread carefully and watch out for low growing shubs with waxy leaves in clusters of three.
\halfPic{}{./images/poison-oak.jpg}{Poison Oak. Don't touch}
\section{How to use this book}
\subsection*{Grades and Descriptions}
As much as posible the grades and descriptions of routes in this book have been based on the collective first hand experience of the collaborators of this book. As much as possible instances where first hand experience is limited or unavailable are aknowledged in the route discription and speculation on grades is avoided when possible. \\
The routes in this book are graded on the Hueco V scale. Although these grades are inherently subjective care has been taken in considering the grading of each route. A color coding system is applied for ease of use as described below.\\
\colorbox{green!20}{\textbf{Routes V0-V3}}\\
\colorbox{RoyalBlue!20}{\textbf{Routes V4-V6}}\\
\colorbox{Goldenrod!50}{\textbf{Routes V7-V9}}\\
\colorbox{red!20}{\textbf{Routes V10+}}\\
\colorbox{black!20}{\textbf{Projects and Unknown Grades}}\\
\subsection*{Ratings for Quality and Seriousness}
In additon to a difficulty rating route quality and seriousness ratings are provided on an out of three system as defined below.\\
\textbf{Quality:}\\
\begin{tabular}{rcp{0.75\linewidth}}
\ding{72} \ding{72} \ding{72}&-&This route is an area classic, if you are unfamiliar with the area this is one you should check out on your first visit.\\
\ding{72} \ding{72}&-&This route is charming, but may be lacking one or more qualities of a true classic.\\
\ding{72}&-&This route may leave something to be desired but isn't objectively terrible.\\
\ding{73}&-&Zero stars, this route is bad.\\
&-&No quality rating given, this designation is typically only included for Projects and routes that the collaborators of this guide do not have first hand knowledge of.\\
\end{tabular}
\textbf{Seriousness:}\\
\begin{tabular}{rcp{0.75\linewidth}}
&-&No seriousness rating given, this is gerneally a safe climb with appropriate padding. There are no extraordinary hazards that you should be aware of.\\
\warn&-&This route has insecure moves which are high off the ground or over a bad landing or both. A competent climber who is aware of these hazards will still be able to climb this at a minimaly increased risk.\\
\warn \warn&-&There are sections of this climb where the risks are hard to minimize. Falls in certain areas may be unlikely for a climber of appropriate skill level but the consequences of such a fall could be real.\\
\warn \warn \warn&-&This route could cause searious injury or worse even when attempted a person competent at climbing the assigned grade. This climb should be approached with caution.\\
\end{tabular}
\subsection*{A Reminder}
The rocks do not know what they are called or how they are supposed to be climbed, likewise your experience does not need to depend on this information. Do not let the descriptions and ratings in this book prevent you from experienceing the rocks as they are.
\clearpage