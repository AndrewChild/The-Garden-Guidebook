\section{Authors Note}
This guidebook is a work in progress which is being periodically updated. Some details have been substituted for placeholder values or omitted entirely. If you would like to contribute your knowledge to this project please get in touch with me (I can be contacted via git hub or my personal email: an.child@gmail.com). Every time this guidebook is updated the latest version is uploaded to git hub, follow the QR code at the end of this section to download the latest version.
\section{Amenities}
\subsection*{Toilets}
Barring emergencies, digging cat holes can be avoided when climbing at the Garden. As an alternative consider driving 1.9 miles back towards Sweet Home to use the pit toilets outside of Sunnyside park. The three minute drive will take roughly the same amount of time as finding a secluded spot and digging a hole and you won't have to worry about squatting on a patch of poison oak. Likewise most of the areas in this book exist within a 5 minutes drive of a toilet or pit toilet, think about it. 
\subsection*{Camping}
Paid campsites can be reserved at Sunny Side Park 1.9 miles away from the Garden Main area. Dispersed camping is allowed on any of the pullouts on Quartzville Creek road East of Green Peter Reservoir. 
\section{Local Ethics}
The Garden Main, Garden Cliffs, and Upper Garden areas are located on private land owned by the Cascade Timber Company. The company allows walk in access to their land, but there is no official relationship between the landowners and climbers. The established ethic for climbing on timber land in Oregon is that the owners prefer not to get involved, consequently climbers should do their best to keep a low-ish profile and ensure the land owners don't need to get involved. There are a few specific activities which could threaten access for everyone:\\
\begin{itemize}
\item Building fires or causing fire hazards.\\
\item Parking on or blocking gated forest roads.\\
\item Overnight camping at the climbing areas.\\
\item Failing to obey posted fire closures.\\
\end{itemize}
\subsection*{Open and Closed Projects}
This book notes several lines that have yet to see a first ascent. Generally boulder projects are understood to be "Open" which is to say that no one has dibs on them and they may be climbed by anyone. These projects are in fact included as a way to encourage and inspire climbers to check out and establish new lines.\\
\\
Rope climbs on the other hand may be either "Open" (anyone can get on it) or "Closed" (the developer has requested others to wait until they are finished establishing the route). Closed projects are customarily marked with a piece of red string or webbing on the first bolt.\\
\begin{center}			  
\begin{overpic}[width=\linewidth]{./images/photos/Cricket.jpg}
\put (0,5) {\colorbox{\chapterColor}{\parbox{0.7\linewidth}{\textcolor{white}{Some of my friends live in the moss.}}}}
\end{overpic}
\end{center}	
There are many reasons why a route developer may choose to "Close" a project. It may be as simple as the developer hasn't finished cleaning and bolting the route. More commonly the developer is just requesting the privilege of the first ascent of the route and the naming rights that come along with it. Route development takes a good deal of time and money (each bolt on a sport climb costs upwards of \$6) thus its considered reasonable for a route developer ask for a period of first dibs on the fruit of their labors. Failure to obey this request is considered route thievery and its not a good way to make friends.\\
\subsection*{We like the moss}
The lush moss coverings that adorn the boulders and cliffs are an essential part of the area's charm. When cleaning boulders and routes try to take a conservative approach and avoid de-mossing unnecessary parts of the rock.
\section{Poison Oak}
The Upper Garden area is plagued by poison oak and it has been seen in patches in other areas as well. Tread carefully and watch out for low growing shrubs with waxy leaves in clusters of three. The leaves turn red during the fall and fall off in the winter. Exposure to any part of the plant can cause irritation.\\
\begin{center}			  
\begin{overpic}[width=\linewidth]{./images/photos/poison-oak.jpg}
\put (0,5) {\colorbox{\chapterColor}{\parbox{0.7\linewidth}{\textcolor{white}{Poison oak}}}}
\end{overpic}
\end{center}			
\newpage
\section{A Timetable of Significant Events}
\subsection*{Prehistory}
\colorbox{Goldenrod!50}{\textbf{c. 16000 BC:}} Humans first travel to North America via a land bridge in the bearing sea.\\
\colorbox{RoyalBlue!20}{\textbf{c. 13000 - 11000 BC:}} The Missoula floods carve out the Willamette Valley.\\
\colorbox{Goldenrod!50}{\textbf{c. 11200 BC:}} Paleo-Indians inhabit caves near Fort Rock. They leave behind the oldest discovered evidence of human activity in Oregon.\\
\colorbox{Goldenrod!50}{\textbf{c. 8000 BC:}} Human settlements are established in the Willamette Valley.\\
\colorbox{Goldenrod!50}{\textbf{c. 6000 BC:}} Petroglyphs are carved into the walls of the Cascadia Cave 6 miles SE of the Garden.\\
\colorbox{RoyalBlue!20}{\textbf{5677 BC:}} Mt. Mazama erupts forming crater lake.\\
\subsection*{1800s}
\colorbox{Goldenrod!50}{\textbf{c. 1850:}} Western settlers first arrive in the Sweet Home Valley.\\
\colorbox{Goldenrod!50}{\textbf{c. 1850:}} Infectious diseases ravage the people of the Kalapuya tribes. Their population dwindles to roughly 600 from 15000 at the beginning of the century.\\
\colorbox{Goldenrod!50}{\textbf{April 12, 1851:}} At the Santiam Treaty Council, Santiam Kalapuya tribal leaders express a strong desire to maintain their territory between the forks of the Santiam River.\\
\colorbox{Goldenrod!50}{\textbf{January 22, 1855:}} The Kalapuya Treaty is signed. The Kalapuya "and other" tribes cede the entire Willamette River drainage to the United States.\\
\colorbox{Goldenrod!50}{\textbf{1861:}} The Santiam Wagon Road is built along a route which closely resembles Highway 20.\\
\colorbox{Goldenrod!50}{\textbf{1893:}} The town of Sweet Home is incorporated.\\
\subsection*{1900s}
\colorbox{green!20}{\textbf{c. 1940:}} Climbing begins at Skinner Butte in Eugene.\\
\colorbox{green!20}{\textbf{1949:}} The Rooster Tail (5.6, the Menagerie) is climbed by Bill Sloan and Byron Taylor (FA).\\
\colorbox{red!20}{\textbf{1959:}} Red Cross Overhang (V9, Jenny Lake) is climbed by John Gill (FA), the first boulder problem of the grade.\\
\colorbox{Goldenrod!50}{\textbf{1962:}} Green Peter Dam is constructed.\\
\colorbox{green!20}{\textbf{1964:}} The South Face of the Santiam Pinnacle (5.6) is climbed by Steve Knutson and Skip King (FA).\\
\colorbox{green!20}{\textbf{c. 1960:}} Legend has it that climbers began recreating at the Garden around this time, some calling it "The Enchanted Forest".\\
\colorbox{Goldenrod!50}{\textbf{1966:}} Foster Dam is constructed.\\
\colorbox{green!20}{\textbf{1968:}} Nicholas A. Dodge publishes "A Climbers Guide to Oregon" the first guidebook for climbing in Oregon.\\
\colorbox{green!20}{\textbf{February 1983:}} Chain Reaction (5.12c, Smith Rock) at Smith Rock is climbed by Alan Watts (FA). Sport climbing is now cool.\\
\colorbox{green!20}{\textbf{1988:}} The first climbing gym in Oregon (Portland Rock Gym) opens.\\
\colorbox{red!20}{\textbf{1991:}} The Hueco "V" scale of boulder grading is created by John Sherman.\\
\colorbox{red!20}{\textbf{1992:}} The first crash pad is sold commercially.\\
\colorbox{green!20}{\textbf{c. 1990:}} Bouldering at the Garden begins in earnest. Many classics such as Octernal (V7), then called Tom's Bad Trip, and the Good (V3) are climbed around now.\\
\colorbox{red!20}{\textbf{1996:}} The Real Thing, a climbing film exclusively featuring bouldering, is released.\\
\colorbox{red!20}{\textbf{1996:}} Chuck Palahniuk publishes "Fight Club".\\
\colorbox{green!20}{\textbf{August 1999:}} Into the Light (V6, the Garden) is climbed by Eric Brown (FA).\\
\colorbox{red!20}{\textbf{1999:}} 8a.nu Is created by Jens Larssen.\\
\subsection*{2000s}
\colorbox{green!20}{\textbf{c. 2000:}} The Upper Garden is logged. Climbers discover new boulders in the clear cut.\\
\colorbox{green!20}{\textbf{September 2003:}} Fight Club (V8, the Garden) is climbed by Craig Malik (FA).\\
\colorbox{red!20}{\textbf{2005:}} Mountain Project is created by Nick Wilder and Andy Laakmann.\\
\colorbox{green!20}{\textbf{April 2006:}} The Garden is first posted on Mountain project.\\
\colorbox{green!20}{\textbf{ January 2013:}} Andrew Child visits the Garden for the first time. He fails to climb Gumby Slab (V1) in the rain.\\
\colorbox{RoyalBlue!20}{\textbf{c. 2015:}} The crimp rail on Octernal (V7, the Garden) breaks forming the iconic lightning bolt hold.\\
\colorbox{red!20}{\textbf{March 2016:}} Horizon (V15, Mt. Hiei) is climbed by Ashima Shiraishi, the first woman, and youngest person to date, to climb the grade.\\
\colorbox{green!20}{\textbf{March 2016:}} Scorpion Hitchhiker's Toilet Bowl Odessy (5.11b, the Garden) climbed by Jayson Nissen (FA), the first free ascent of a bolted route at the Garden Cliffs.\\
\colorbox{green!20}{\textbf{July 2016:}} The Quartzville Creek boulders are established.\\
\colorbox{red!20}{\textbf{23 October, 2016:}} Burden of Dreams (V17, Lapnor) is climbed by Nalle Hukkataival (FA), the first boulder problem of the grade.\\
\colorbox{green!20}{\textbf{June 2019:}} The Pink Tag Boulders are established.\\
\colorbox{red!20}{\textbf{January 2021:}} Mountain Project is acquired by onXmaps, Inc.\\
\colorbox{green!20}{\textbf{July 2022:}} Andrew Child, begins working on this guidebook.\\
\colorbox{RoyalBlue!20}{\textbf{June 2023:}} The crimp rail on Octernal (V7, the Garden) breaks again. The lightning bolt hold is no more.\\
\colorbox{green!20}{\textbf{2023:}} The Willamette Area Climbers Coalition is founded.\\
\colorbox{green!20}{\textbf{October 2025:}} The first bolted sport climb, Freedom of the Press, is established at the upper Garden Cliffs.\\
\begin{center}			  
\begin{overpic}[width=\linewidth]{./images/photos/petes.JPG}
\put (0,5) {\colorbox{\chapterColor}{\parbox{0.7\linewidth}{\textcolor{white}{Eric Bailey on Pete's Rail circa 2002.}}}}
\end{overpic}
\end{center}		
\newpage
\section{How to use this book}
\subsection*{Route names}
It isn't uncommon for a section of stone at the Garden to go by multiple different names. In such cases this guidebook seeks call a route whatever name is currently most commonly used (usually the name a route is given on Mountain Project). Other names are often noted in the routes description. In cases where a route has no known name a name has been made up for this guidebook (these route names are marked with an asterisk).
\subsection*{Grades and Descriptions}
As much as possible the grades and descriptions of routes in this book have been based on the collective first hand experience of the collaborators of this book. Instances where first hand experience is limited or unavailable are graded with an asterisk.\\
\\
Boulder problems in this book are graded on the Hueco V scale and roped climbs are graded using the Yosemite decimal system. Although these grades are inherently subjective, care has been taken in considering the grading of each route. A color coding system is applied for ease of use as described below.\\
\newline
\colorbox{green!20}{\textbf{Boulder problems V0-V3}}\\
\colorbox{RoyalBlue!20}{\textbf{Boulder problems V4-V6}}\\
\colorbox{Goldenrod!50}{\textbf{Boulder problems V7-V9}}\\
\colorbox{red!20}{\textbf{Boulder problems V10+}}\\
\colorbox{green!20}{\textbf{Roped climbs 5.0-5.9}}\\
\colorbox{RoyalBlue!20}{\textbf{Roped climbs 5.10a-5.11d}}\\
\colorbox{Goldenrod!50}{\textbf{Roped climbs 5.12a-5.13d}}\\
\colorbox{red!20}{\textbf{Roped climbs 5.14a+}}\\
\colorbox{black!20}{\textbf{Projects and Unknown Grades}}\\
\subsection*{Ratings for Quality and Seriousness}
In addition to a difficulty rating, route quality and seriousness ratings are provided on an out of three system as defined below.
\subsubsection{Quality}
\begin{tabular}{rcp{0.75\linewidth}}
&-&No quality rating given, this designation is typically only included for Projects and routes that the collaborators of this guide do not have first hand knowledge of.\\
\ding{72} \ding{72} \ding{72}&-&This route is an area classic, if you are unfamiliar with the area this is one you should check out on your first visit.\\
\ding{72} \ding{72}&-&This route is charming, but may be lacking one or more qualities of a true classic.\\
\ding{72}&-&This route may leave something to be desired but isn't objectively terrible.\\
\ding{73}&-&Zero stars, this route is bad.\\
\end{tabular}\\
\subsubsection{Seriousness}
\begin{tabular}{rcp{0.75\linewidth}}
&-&No seriousness rating given, this is generally a safe climb with appropriate padding/protection. There are no extraordinary hazards that you should be aware of.\\
\warn&-&A boulder with this rating may have insecure moves which are high off the ground or over a bad landing or both. A roped climb with this rating may have sections were falling presents risk of injury. A competent climber who is aware of these hazards will still be able to climb this at a minimally increased risk.\\
\warn \warn&-&There are sections of this climb where the risks are hard to minimize. Falls in certain areas may be unlikely for a climber of appropriate skill level but the consequences of such a fall could be real.\\
\warn \warn \warn&-&This route could cause serious injury or worse even when attempted by a person competent at climbing the assigned grade. This climb should be approached with caution.\\
\end{tabular}
\subsection*{A Reminder}
The rocks do not know what they are called or how they are supposed to be climbed, likewise your experience does not need to depend on this information. Do not let the descriptions, grades, and ratings assigned in this book prevent you from experiencing the rocks as they are.