\formatChapter{Armageddon}
\raggedcolumns
\begin{multicols}{2}
			\fullPic{./maps/area/upperGarden_c.png}

	\qrcode{./maps/qr//Armageddon_qr.png}{http://maps.google.com/maps?q=44.43959094940084,-122.58215256842753}{Navigate to this area}
Located about 3.2 miles down quatzville road from highway 20, park in the Gravel pull out where the road bends about 0.1 miles before you reach a left hand turnoff to a gated logging road (MS-310). Follow the logging road approximately 200 yards up hill until it veers slightly to the right. Look for a trail that cuts right through a thin patch of trees to the boulder field (Note: there are a couple of trails and its worth getting on the most tred one as the others are unpleasent). The lack of shade, the blackberries, the poison oak (holy shit there is a lot of poison oak), and the 3 minute approach all make this area less desirable and far less traveled then the Main. This area is also known as the upper garden.
\newline
\textbf{NOTE: This area is mostly incomplete. Look forward to more information in future revisions of this book or contribute your own knowledge on github.}
\halfPic{./maps/plots//Armageddon.png}{}

\newpage
				\halfPic{./maps/area/entranceUpper.png}{Entrance area map}

		\section{A - Entrance Area}\label{sa:Entrance Area}
	\begin{minipage}{\columnwidth}
	\
	\end{minipage}
	
			\begin{minipage}{\columnwidth}
			\subsection*{Pumpkin Boulder}\label{bf:Pumpkin Boulder}
			This is the first boulder that you encounter when approaching the area.
			
			\end{minipage}
			
					\begin{minipage}{\linewidth}	
					\label{rt:Pumpkin Project}
\colorbox{black!20}{
\parbox{0.95\textwidth}{
\textbf{
1 Pumpkin Project V?  
}
}
}

					\begin{adjustwidth}{0.5cm}{}				
					Quality line on the uphill side of the boulder seems like it will go at around V7. Certainly possible that this has been done before.
						\newline (No Topo) 
					\end{adjustwidth}
					\end{minipage}
			\begin{minipage}{\columnwidth}
			\subsection*{Baseball Boulder}\label{bf:Baseball Boulder}
			This is one of the few boulders that isn't covered in poison oak, but there is quite a lot of it sounding it. Approach with caution.
			
			\end{minipage}
			
								\halfPic{./maps/topos/baseball_c.png}{Baseball}

					\begin{minipage}{\linewidth}	
					\label{rt:Baseball}
\colorbox{green!20}{
\parbox{0.95\textwidth}{
\textbf{
2 Baseball V3- \ding{72}  
}
}
}

					\begin{adjustwidth}{0.5cm}{}				
					Sit start with a high left hand on a good dish around the blunt corner and a low right hand pinch. Pull a powerful move to good edges and continue straight up.
					\end{adjustwidth}
					\end{minipage}
					\begin{minipage}{\linewidth}	
					\label{rt:Bunt}
\colorbox{green!20}{
\parbox{0.95\textwidth}{
\textbf{
3 Bunt V1 \ding{72}  
}
}
}

					\begin{adjustwidth}{0.5cm}{}				
					Sit start with both hands in a low bubbly pod. Climb straight up.
					\end{adjustwidth}
					\end{minipage}
\newpage
				\fullPic{./maps/area/bread.png}

		\section{B - The Bread Loaves/Scratch and Spliff}\label{sa:The Bread Loaves/Scratch and Spliff}
	\begin{minipage}{\columnwidth}
	These two boulders are the area's main attraction. Historically some groups have called both boulders Scratch and Spliff while others called them both the Bread Loaves. The modern compromise seems to be that the upper boulder is Scratch and Spliff while the lower boulder is the Bread Loaf.
	\end{minipage}
	
			\begin{minipage}{\columnwidth}
			\subsection*{Bread Loaf}\label{bf:Bread Loaf}
			\
			
			\end{minipage}
			
								\halfPic{./maps/topos/breadLoaf_c.png}{Bread Loaf}

					\begin{minipage}{\linewidth}	
					\label{rt:Buddha's Belly}
\colorbox{RoyalBlue!20}{
\parbox{0.95\textwidth}{
\textbf{
1 Buddha's Belly V4 \ding{72} \ding{72}  
}
}
}

					\begin{adjustwidth}{0.5cm}{}				
					Stand start on two horizontal edges. Navigate your way to some good lumpy jugs midway up the route and either mantle or side pull your way to the top. Also called bread loaf left.
					\end{adjustwidth}
					\end{minipage}
					\begin{minipage}{\linewidth}	
					\label{rt:Bread Loaf Traverse}
\colorbox{RoyalBlue!20}{
\parbox{0.95\textwidth}{
\textbf{
2 Bread Loaf Traverse V5 \ding{72} \ding{72}  
}
}
}

					\begin{adjustwidth}{0.5cm}{}				
					stand start with hands matched in the left of two good pods in the lowest diagonal crack. Follow the crack system right with the help of a good hold under the roof. top along the arete. Dabby.
					\end{adjustwidth}
					\end{minipage}
						\begin{adjustwidth}{0.5cm}{}				
						\textbf{Variations:} \newline
							\begin{minipage}{\linewidth}	
							\label{vr:Baker's Dozen}
\colorbox{Goldenrod!50}{
\parbox{0.95\textwidth}{
\textbf{
2a Baker's Dozen V8*  
}
}
}

							\begin{adjustwidth}{0.5cm}{}				
							Start as for Buddha's Belly, traverse into the bread loaf traverse.
							\end{adjustwidth}
							\end{minipage}
						\end{adjustwidth}
								\halfPic{./maps/topos/breadLoaf2_c.png}{Bread Loaf 2}

					\begin{minipage}{\linewidth}	
					\label{rt:Worf}
\colorbox{RoyalBlue!20}{
\parbox{0.95\textwidth}{
\textbf{
3 Worf V5 \ding{72} \ding{72}  
}
}
}

					\begin{adjustwidth}{0.5cm}{}				
					Starting from two horizontal cracks a bizarre sequence leads you first left then right as you climb the rounded corner. Some but not all of the difficulty comes from the dab potential.
					\end{adjustwidth}
					\end{minipage}
						\fullPic{./maps/topos/scratchSpliff2_c.png}

			\begin{minipage}{\columnwidth}
			\subsection*{Scratch and Spliff}\label{bf:Scratch and Spliff}
			\
			
			\end{minipage}
			
					\begin{minipage}{\linewidth}	
					\label{rt:Scratch and Spliff Traverse}
\colorbox{green!20}{
\parbox{0.95\textwidth}{
\textbf{
4 Scratch and Spliff Traverse V3 \ding{72} \ding{72} \ding{72}  
}
}
}

					\begin{adjustwidth}{0.5cm}{}				
					Start at the far right of the major horizontal crack (as for Roach) and traverse all the way left topping out along a juggy vertical crack system.
					\end{adjustwidth}
					\end{minipage}
						\begin{adjustwidth}{0.5cm}{}				
						\textbf{Variations:} \newline
							\begin{minipage}{\linewidth}	
							\label{vr:Late Start}
\colorbox{green!20}{
\parbox{0.95\textwidth}{
\textbf{
4a Late Start* V2 \ding{72} \ding{72}  
}
}
}

							\begin{adjustwidth}{0.5cm}{}				
							Sit start with juggy holds at the top of a low ramp. Climb straight up into the top of Scratch and Spliff Traverse.
							\end{adjustwidth}
							\end{minipage}
						\end{adjustwidth}
					\begin{minipage}{\linewidth}	
					\label{rt:Scratch}
\colorbox{RoyalBlue!20}{
\parbox{0.95\textwidth}{
\textbf{
5 Scratch V4 \ding{72} \ding{72}  
}
}
}

					\begin{adjustwidth}{0.5cm}{}				
					Stand start with right hand on a good hold in the horizontal crack and left hand wrapping around a juggy corner. Jump to a bubbly rail and tick tack your way to the top. Originally this route started as for Scratch and Spliff Traverse.
					\end{adjustwidth}
					\end{minipage}
								\halfPic{./maps/topos/scratchSpliff_c.png}{Scratch and Spliff}

					\begin{minipage}{\linewidth}	
					\label{rt:Spliff}
\colorbox{green!20}{
\parbox{0.95\textwidth}{
\textbf{
6 Spliff V3 \ding{72} \ding{72} \ding{72}  \warn 
}
}
}

					\begin{adjustwidth}{0.5cm}{}				
					Start on a large hanging flake. Climb straight up. Sit start seems possible but wouldn't add much to the experience.
					\end{adjustwidth}
					\end{minipage}
					\begin{minipage}{\linewidth}	
					\label{rt:Roach}
\colorbox{green!20}{
\parbox{0.95\textwidth}{
\textbf{
7 Roach V0 \ding{72} \ding{72}  
}
}
}

					\begin{adjustwidth}{0.5cm}{}				
					Stand start with a good edge in the horizantal crack..
					\end{adjustwidth}
					\end{minipage}
					\begin{minipage}{\linewidth}	
					\label{rt:For What It's Worth}
\colorbox{green!20}{
\parbox{0.95\textwidth}{
\textbf{
8 For What It's Worth* V2 \ding{72} \ding{72}  
}
}
}

					\begin{adjustwidth}{0.5cm}{}				
					Squat start on a low ramp on the NW corner of the boulder using a left hand low on the arete and a right hand side pull. Bump up the arete then Dyno to the lip. Dab potential creates a lot of the difficulty.
					\end{adjustwidth}
					\end{minipage}
								\halfPic{./maps/topos/scratchSpliff3_c.png}{Scratch and Spliff 3}

					\begin{minipage}{\linewidth}	
					\label{rt:Caliban's War}
\colorbox{RoyalBlue!20}{
\parbox{0.95\textwidth}{
\textbf{
9 Caliban's War V6*  
}
}
}

					\begin{adjustwidth}{0.5cm}{}				
					Stand start with hand holds in a horizontal crack. Crank one move to the lip.
					\end{adjustwidth}
					\end{minipage}
					\begin{minipage}{\linewidth}	
					\label{rt:Stoned Age}
\colorbox{green!20}{
\parbox{0.95\textwidth}{
\textbf{
10 Stoned Age V2*  
}
}
}

					\begin{adjustwidth}{0.5cm}{}				
					It looks like you could easily climb from the horizontal crack to a diagonal crack on the upper right, but the landing is very poor. Older guidebooks indicate that this has been done.
					\end{adjustwidth}
					\end{minipage}
\newpage
		\section{C - Dr. Strangelove Area}\label{sa:Dr. Strangelove Area}
	\begin{minipage}{\columnwidth}
	\
	\end{minipage}
	
			\begin{minipage}{\columnwidth}
			\subsection*{Dr. Strange Love}\label{bf:Dr. Strange Love}
			\
			
			\end{minipage}
			
\end{multicols}
\clearpage