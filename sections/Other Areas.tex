\thispagestyle{empty}
\colorlet{shadecolor}{\chapterColor}
\chapter{Other Areas}
\fancyhead{}
\lhead[\textcolor{\chapterColor}{\rule[-2pt]{\textwidth}{15pt}}]{\textcolor{\chapterColor}{\rule[-2pt]{\textwidth}{15pt}}\hspace{-\textwidth}\color{white}\hspace{4pt}\protect\thepage\hspace{1ex}-\hspace{1ex}Other Areas}
\rhead[\textcolor{\chapterColor}{\rule[-2pt]{\textwidth}{15pt}}\hspace{-\textwidth}\color{white}Other Areas \protect\thepage \hspace{4pt}]{\textcolor{\chapterColor}{\rule[-2pt]{\textwidth}{15pt}}}
\fancyhead[RO]{}
\fancyhead[RE]{\color{white}Other Areas\hspace{1ex}-\hspace{1ex}\protect\thepage \hspace{4pt}}
\raggedcolumns
\begin{multicols}{2}
The following areas are nearby Sweet Home and the Garden. They are included here in limited detail either due to their obscurity, the author's lack of familiarity, or both. These areas are roughly organized in descending order of their distance to the Garden.\\
\needspace{6em}
\section{A - Under Developed Garden Subareas}\phantomsection\label{sa:Under Developed Garden Subareas}
All of these spots are within the "boarders" of the garden, but they have not been developed or were once developed and are now overgrown.\\
\needspace{10em}
\subsection*{The Erb Garden}\phantomsection\label{bf:The Erb Garden}
There are several boulders in the vegetation past E's Dirty B. Some of these have been climbed in the past but this spot has mostly been passed over as lacking potential. See index for GPS location.\\
\phantomsection\label{pt:MS 300}
	\setbox0=\hbox{\begin{overpic}[width=0.8\linewidth]{./images/photos/MS300.jpg}\put (0,5) {
\colorbox{\chapterColor}{
\parbox{0.8\linewidth-0.1\linewidth}{
\textcolor{white}{
MS300 Boulders
}}}}
	\end{overpic}}
	\needspace{\ht0}
	\begin{center}
	\begin{overpic}[width=0.8\linewidth]{./images/photos/MS300.jpg}\put (0,5) {
\colorbox{\chapterColor}{
\parbox{0.8\linewidth-0.1\linewidth}{
\textcolor{white}{
MS300 Boulders
}}}}
	\end{overpic}
	\end{center}
\needspace{10em}
\subsection*{MS300 Boulders}\phantomsection\label{bf:MS300 Boulders}
There are a few good looking boulders on the forest road just uphill of the Main area. Unfortunately they are in a massive patch of blackberries. Its possible that these were climbed long ago. See index for GPS location.\\
\needspace{10em}
\subsection*{Upper Garden Cliffs}\phantomsection\label{bf:Upper Garden Cliffs}
The Upper Garden is bordered by an extensive cliff. Unfortunately in many sections the area around the cliff and the cliff itself are both covered in poison oak. A free rope adorns the cliff above the Dr. Strangelove boulder for anyone who wants it. Said rope exists as the only remnants of a would be developer who found himself surrounded by poison oak and gave up, never to return.\\\\
\\\\
See approach instructions for the Upper Garden. See index for GPS location.\\
\needspace{10em}
\subsection*{Upper Forest Cliff}\phantomsection\label{bf:Upper Forest Cliff}
The far west end of the upper garden cliff band has recently seen some exploration. Presently there is a rough trail along the base and a single bolted sport route. See index for GPS location.\\
\needspace{2em}
\phantomsection\label{rt:Freedom of the Press}
\colorbox{RoyalBlue!20}{
\parbox{0.95\linewidth}{
\hspace{-1ex}\textbf{$\Box$
1 Freedom of the Press 5.10a \ding{72}\ding{72} 
}}}
\begin{adjustwidth}{1.3em}{}			
Sport, 4 bolts. Climbs a shallow dihedral using stems to a tenuous finish. Still exfoliating.
  (No Topo)
\end{adjustwidth}
\newpage
\section{B - Highway 11 Crags}\phantomsection\label{sa:Highway 11 Crags}
There are a few forgotten sport climbs further down the road beyond the Garden.\\
\textbf{NOTE: Warning the routes on these feature are equipped with "cold shut" bolts. These bolts are often not strong enough to hold weight and should be avoided or only used with extreme caution.}\\
\needspace{10em}
\subsection*{Highway 11 Wall}\phantomsection\label{bf:Highway 11 Wall}
\setbox0=\hbox{\includegraphics[width=0.4\linewidth]{./images/maps/qr//Highway 11 Wall_qr.png}}% Store image in \box0
\needspace{\ht0}% Need at least the height of \box0
\begin{minipage}[c]{0.5\linewidth}{
\includegraphics[width=0.8\linewidth]{./images/maps/qr//Highway 11 Wall_qr.png}}
\end{minipage}
\begin{minipage}[c]{0.5\linewidth}{
\textcolor{blue}{\parbox{0.8\linewidth}{\href{http://maps.google.com/maps?q=44.45543296270457,-122.54719340682593}{\ul{Navigate to this formation}}}}}
\end{minipage}
\\
\\
There are several bolted routes on this unappealing highway cut. Per the note below the bolts here are ancient and of suspicious construction. Do yourself a favor and go somewhere else.\\
\needspace{10em}
\subsection*{Highway 11 Pillar}\phantomsection\label{bf:Highway 11 Pillar}
\setbox0=\hbox{\includegraphics[width=0.4\linewidth]{./images/maps/qr//Highway 11 Pillar_qr.png}}% Store image in \box0
\needspace{\ht0}% Need at least the height of \box0
\begin{minipage}[c]{0.5\linewidth}{
\includegraphics[width=0.8\linewidth]{./images/maps/qr//Highway 11 Pillar_qr.png}}
\end{minipage}
\begin{minipage}[c]{0.5\linewidth}{
\textcolor{blue}{\parbox{0.8\linewidth}{\href{http://maps.google.com/maps?q=44.50089,-122.47419}{\ul{Navigate to this formation}}}}}
\end{minipage}
\\
\\
A ~40' knob on the side of the road hosts few lines. For more information on this area consult "Rock Climbing Oregon" by  Adam R. Bolf and Benjamin P. Ruef.\\
\needspace{6em}
\phantomsection\label{pt:Cascadia}
	\setbox0=\hbox{\begin{overpic}[width=0.9\linewidth]{./images/photos/CascadiaCave2.jpg}\put (0,5) {
\colorbox{\chapterColor}{
\parbox{0.9\linewidth-0.1\linewidth}{
\textcolor{white}{
This is a cultural site. Do not climb here.
}}}}
	\end{overpic}}
	\needspace{\ht0}
	\begin{center}
	\begin{overpic}[width=0.9\linewidth]{./images/photos/CascadiaCave2.jpg}\put (0,5) {
\colorbox{\chapterColor}{
\parbox{0.9\linewidth-0.1\linewidth}{
\textcolor{white}{
This is a cultural site. Do not climb here.
}}}}
	\end{overpic}
	\end{center}
\section{C - Cascadia Cave}\phantomsection\label{sa:Cascadia Cave}
Do not climb here. This cliff is located near by the Garden and is the site of "one of the most significant cultural resources of the Indigenous peoples of the Cascade Mountains and the Willamette Valley". It contains petroglyphs which are 8,000 years old and is a spiritual location for the local Kalapuya and Molala tribes. It's also not that big and doesn't have much climbing potential, so maybe we should just leave this one be. \\\\
\\\\
It is included in this section because there is no signage at the cliff which would alert a visitor to its significance. For obvious reasons the location of this area has been omitted from this document, but it's not that hard to find if you wish to visit this place for reasons other than rock climbing development.\\
\vfill\null
\columnbreak
\section{D - Canyon Creek}\phantomsection\label{sa:Canyon Creek}
\setbox0=\hbox{\includegraphics[width=0.4\linewidth]{./images/maps/qr//Canyon Creek_qr.png}}% Store image in \box0
\needspace{\ht0}% Need at least the height of \box0
\begin{minipage}[c]{0.5\linewidth}{
\includegraphics[width=0.8\linewidth]{./images/maps/qr//Canyon Creek_qr.png}}
\end{minipage}
\begin{minipage}[c]{0.5\linewidth}{
\textcolor{blue}{\parbox{0.8\linewidth}{\href{http://maps.google.com/maps?q=44.39708529718213,-122.44671253776127}{\ul{Navigate to this sub area}}}}}
\end{minipage}
\\
\\
There's a little cliff band and a few boulders on the beach at the confluence of Canyon Creek and the Lower Santiam River.\\\\
\\\\
Located on Highway 20 10 miles east of Quartzville Road. Park in the dirt pullout on the side of the road just east of concrete bridge and across from Canyon Creek Road. Follow feint trails to the river and watch out for poison oak.\\
\needspace{6em}
\section{E - Horse Rock}\phantomsection\label{sa:Horse Rock}
\setbox0=\hbox{\includegraphics[width=0.4\linewidth]{./images/maps/qr//Horse Rock_qr.png}}% Store image in \box0
\needspace{\ht0}% Need at least the height of \box0
\begin{minipage}[c]{0.5\linewidth}{
\includegraphics[width=0.8\linewidth]{./images/maps/qr//Horse Rock_qr.png}}
\end{minipage}
\begin{minipage}[c]{0.5\linewidth}{
\textcolor{blue}{\parbox{0.8\linewidth}{\href{http://maps.google.com/maps?q=44.31343,-122.33943}{\ul{Navigate to this sub area}}}}}
\end{minipage}
\\
\\
A cool tower in the woods with at least one route on it. There are more formations nearby which also hold potential.\\
\vfill\null
\columnbreak
\section{F - Gordon Ridge}\phantomsection\label{sa:Gordon Ridge}
\setbox0=\hbox{\includegraphics[width=0.4\linewidth]{./images/maps/qr//Gordon Ridge_qr.png}}% Store image in \box0
\needspace{\ht0}% Need at least the height of \box0
\begin{minipage}[c]{0.5\linewidth}{
\includegraphics[width=0.8\linewidth]{./images/maps/qr//Gordon Ridge_qr.png}}
\end{minipage}
\begin{minipage}[c]{0.5\linewidth}{
\textcolor{blue}{\parbox{0.8\linewidth}{\href{http://maps.google.com/maps?q=44.34844,-122.34942}{\ul{Navigate to this sub area}}}}}
\end{minipage}
\\
\\
Sparse deposits solid rock dots the landscape high above the Santiam along Gordon Ridge. This region was developed in the early 2010s but it never caught on in the same way that the Garden did and today very little evidence of climbing remains. The center of this previous wave of development was an area called "The Vines". Being a talus field located at high elevation (4000') this area probably has a slightly different season than the Garden. Any aspiring archaeologists looking to rediscover this region can reach out to the author of this book for more information.\\
\needspace{6em}
\section{G - The Menagerie Wilderness}\phantomsection\label{sa:The Menagerie Wilderness}
\setbox0=\hbox{\includegraphics[width=0.4\linewidth]{./images/maps/qr//The Menagerie Wilderness_qr.png}}% Store image in \box0
\needspace{\ht0}% Need at least the height of \box0
\begin{minipage}[c]{0.5\linewidth}{
\includegraphics[width=0.8\linewidth]{./images/maps/qr//The Menagerie Wilderness_qr.png}}
\end{minipage}
\begin{minipage}[c]{0.5\linewidth}{
\textcolor{blue}{\parbox{0.8\linewidth}{\href{http://maps.google.com/maps?q=44.42210107163126,-122.31253942150109}{\ul{Navigate to this sub area}}}}}
\end{minipage}
\\
\\
The Menagrie is one of Oregon's strangest and most mysterious climbing areas. A large valley is decorated by several freestanding towers. For more information on this legendary crag seek out "Rock Climbing Western Oregon: Willamette" by Greg Orton.\\
\needspace{6em}
\phantomsection\label{pt:pinnacle}
	\setbox0=\hbox{\begin{overpic}[width=1.0\linewidth]{./images/photos/pinnacle.jpg}\put (0,5) {
\colorbox{\chapterColor}{
\parbox{1.0\linewidth-0.1\linewidth}{
\textcolor{white}{
Santiam Pinnacle as seen from the road.
}}}}
	\end{overpic}}
	\needspace{\ht0}
	\begin{center}
	\begin{overpic}[width=1.0\linewidth]{./images/photos/pinnacle.jpg}\put (0,5) {
\colorbox{\chapterColor}{
\parbox{1.0\linewidth-0.1\linewidth}{
\textcolor{white}{
Santiam Pinnacle as seen from the road.
}}}}
	\end{overpic}
	\end{center}
\section{H - Santiam Pinnacle}\phantomsection\label{sa:Santiam Pinnacle}
\setbox0=\hbox{\includegraphics[width=0.4\linewidth]{./images/maps/qr//Santiam Pinnacle_qr.png}}% Store image in \box0
\needspace{\ht0}% Need at least the height of \box0
\begin{minipage}[c]{0.5\linewidth}{
\includegraphics[width=0.8\linewidth]{./images/maps/qr//Santiam Pinnacle_qr.png}}
\end{minipage}
\begin{minipage}[c]{0.5\linewidth}{
\textcolor{blue}{\parbox{0.8\linewidth}{\href{http://maps.google.com/maps?q=44.38945069565184,-122.19593395548975}{\ul{Navigate to this sub area}}}}}
\end{minipage}
\\
\\
The Santiam Pinnacle is a rock tower hidden in the tree's just off of Highway 20. It hosts a worthwhile 5.6 multipitch as well as a few other more obscure offerings. Rumor has it that this area has seen recent development.\\\\
\\\\
Located on Highway 20 27 miles from Quartzville Road. Park in a small pullout on the south side of the road and look for a trail heading into the trees on the north side of the road.\\
\newpage
\section{I - Iron Mountain}\phantomsection\label{sa:Iron Mountain}
Iron mountain is a well known hiking trail at the summit of tombstone pass. Although it's iconic spire has been climbed it does not see regular traffic for the obvious reason that it is a dangerous choss pile. The greater area has also been identified as having bouldering potential. To date only a hand full of boulders are known to have been developed.\\
\phantomsection\label{tp:Big Iron}
	\setbox0=\hbox{\begin{overpic}[width=0.9\linewidth]{./images/maps/topos/Other/bigIron_c.png}
	\end{overpic}}
	\needspace{\ht0}
	\begin{center}
	\begin{overpic}[width=0.9\linewidth]{./images/maps/topos/Other/bigIron_c.png}
	\end{overpic}
	\end{center}
\needspace{10em}
\subsection*{Big Iron}\phantomsection\label{bf:Big Iron}
This humble boulder may have less to offer than other areas documented in this book, but since it is not far from the road it may be a nice stop for those passing through on their way across the Cascades.\\\\
\\\\
Located on NF-15 approximately 0.8 miles south of highway 20. Look for the boulder amidst a forested cluster of smaller boulders about 50' off the road on the downhill side. See index for GPS location.\\
\needspace{2em}
\phantomsection\label{rt:Big Iron}
\colorbox{RoyalBlue!20}{
\parbox{0.95\linewidth}{
\hspace{-1ex}\textbf{$\Box$
1 Big Iron V5 \ding{72}\ding{72} 
}}}
\begin{adjustwidth}{1.3em}{}			
Sit start under the roof in compression with left hand on a sloper rail and right hand pinching a small right angle corner. Crank a few techy moves to a juggy top out. Take caution, the blocky holds on the upper half were solid at the time of this route's development in 2023, but there's no telling what a few years of freeze thaw cycles will do to them.
\end{adjustwidth}
\begin{adjustwidth}{0.5cm}{}				
\needspace{4em}
\needspace{2em}
\phantomsection\label{vr:Big Iron Direct}
\colorbox{RoyalBlue!20}{
\parbox{0.95\linewidth}{
\hspace{-1ex}\textbf{$\Box$
1a Big Iron Direct V5 \ding{72}\ding{72} 
}}}
\begin{adjustwidth}{1.3em}{}			
Start as for Big Iron, but top to the left of the roof.
\end{adjustwidth}
\end{adjustwidth}
\needspace{2em}
\phantomsection\label{rt:No Stone Unturned}
\colorbox{green!20}{
\parbox{0.95\linewidth}{
\hspace{-1ex}\textbf{$\Box$
2 No Stone Unturned V2 \ding{72} 
}}}
\begin{adjustwidth}{1.3em}{}			
Sit start with hands matched on a triangle shaped ledge. Climb up and right while trying not to dab.
\end{adjustwidth}
\phantomsection\label{tp:Wild Roses}
	\setbox0=\hbox{\begin{overpic}[width=0.9\linewidth]{./images/maps/topos/Other/wildRoses_c.png}
	\end{overpic}}
	\needspace{\ht0}
	\begin{center}
	\begin{overpic}[width=0.9\linewidth]{./images/maps/topos/Other/wildRoses_c.png}
	\end{overpic}
	\end{center}
\needspace{2em}
\phantomsection\label{rt:Wild Roses}
\colorbox{green!20}{
\parbox{0.95\linewidth}{
\hspace{-1ex}\textbf{$\Box$
3 Wild Roses V2 \ding{72}\ding{72} 
}}}
\begin{adjustwidth}{1.3em}{}			
Squat start with hands matched on a double pinch feature at the center of of the lichen covered scoop. Trending right out of the scoop is a little easier than staying left. Giving this route two stars may be generous.
\end{adjustwidth}
\newpage
	\end{multicols}
\phantomsection\label{pt:needles}
	\setbox0=\hbox{\begin{overpic}[width=1.0\linewidth]{./images/photos/needles.jpg}\put (0,5) {
\colorbox{\chapterColor}{
\parbox{1.0\linewidth-0.1\linewidth}{
\textcolor{white}{
The Needles as seen from NF-1133.
}}}}
	\end{overpic}}
	\needspace{\ht0}
	\begin{center}
	\begin{overpic}[width=1.0\linewidth]{./images/photos/needles.jpg}\put (0,5) {
\colorbox{\chapterColor}{
\parbox{1.0\linewidth-0.1\linewidth}{
\textcolor{white}{
The Needles as seen from NF-1133.
}}}}
	\end{overpic}
	\end{center}
	\begin{multicols}{2}
\section{J - The Needles}\phantomsection\label{sa:The Needles}
\setbox0=\hbox{\includegraphics[width=0.4\linewidth]{./images/maps/qr//The Needles_qr.png}}% Store image in \box0
\needspace{\ht0}% Need at least the height of \box0
\begin{minipage}[c]{0.5\linewidth}{
\includegraphics[width=0.8\linewidth]{./images/maps/qr//The Needles_qr.png}}
\end{minipage}
\begin{minipage}[c]{0.5\linewidth}{
\textcolor{blue}{\parbox{0.8\linewidth}{\href{http://maps.google.com/maps?q=44.59672447306725,-122.15597149494864}{\ul{Navigate to this sub area}}}}}
\end{minipage}
\\
\\
The Needles is a large formation of basalt pillars located high in the hills south of Detroit Lake. The walls are around 100' tall in some sections.\\\\
\\\\
Andrew Child and Alex Funk went on an expedition to this crag in the Summer of 2018. They climbed a few routes on the main face and a spire at the top of the wall, which they later learned was the highest point on the formation. The area has also apparently been explored for its bouldering potential as it is listed in the book Portland Bouldering as the "Lucky Boulders", not much detail is provided. The rock isn't as solid as the garden cliffs but it is of good enough quality for climbing, there is substantial potential for adventure trad climbing here. It is possible that this crag was adversely effected by the 2020 wildfires. If you would like to explore this area feel free to reach out to the author of this guide for more information.\\
	\end{multicols}
\clearpage