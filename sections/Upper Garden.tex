














\formatChapter{Upper Garden}


	\setbox0=\hbox{\begin{overpic}[width=0.8\linewidth]{./maps/area/out/upperGarden_c.png}
	\end{overpic}}
	\needspace{\ht0}
	\begin{center}
	\begin{overpic}[width=0.9\linewidth]{./maps/area/out/upperGarden_c.png}\label{am:Armageddon Area Overview}
	\end{overpic}
	\end{center}


\raggedcolumns
\begin{multicols}{2}
\qrcode{./maps/qr//Upper Garden_qr.png}{http://maps.google.com/maps?q=44.43959094940084,-122.58215256842753}{Navigate to this area}
\includegraphics[width=\linewidth]{./maps/plots//Upper Garden.png}
\end{multicols}
\begin{multicols}{2}



Located about 3.2 miles down quatzville road from highway 20, park in the Gravel pull out where the road bends about 0.1 miles before you reach a left hand turnoff to a gated logging road (MS-310). Follow the logging road approximately 200 yards up hill until it veers slightly to the right. Look for a trail that cuts right through a thin patch of trees to the boulder field (Note: there are a couple of trails and its worth getting on the most tread one as the others are unpleasant). This area is also known as Armageddon or The Clear Cut.

Although the Upper Garden appears overgrown, this entire area was clear cut in the early 2000s. Following the clear cut there was almost no vegitation in the area and it was realatively easy to approach and develop the Upper Gardens many boulders. This is why the photos in this guide look dramatically different from photos in guides from that era. Even the most isolated and overgrown boulders in this area already have names and routes on them, many of these such boulders have been omitted from this guide since they have been swallowed by the poison oak.\\

\textbf{NOTE: This area is mostly incomplete. Look forward to more information in future revisions of this book or contribute your own knowledge on github.}\\


\newpage
	\setbox0=\hbox{\begin{overpic}[width=0.8\linewidth]{./maps/area/out/entranceUpper_c.png}
	\end{overpic}}
	\needspace{\ht0}
	\begin{center}
	\begin{overpic}[width=0.9\linewidth]{./maps/area/out/entranceUpper_c.png}\label{sm:Entrance area map}
	\end{overpic}
	\end{center}


\section{A - Entrance Area}\label{sa:Entrance Area}



	\setbox0=\hbox{\begin{overpic}[width=0.8\linewidth]{./maps/topos/pSpice_c.png}
	\end{overpic}}
	\needspace{\ht0}
	\begin{center}
	\begin{overpic}[width=0.9\linewidth]{./maps/topos/pSpice_c.png}\label{tp:Pumpkin}
	\end{overpic}
	\end{center}


\needspace{1.5cm}
\subsection*{Pumpkin}\label{bf:Pumpkin}
This is the first boulder that you encounter when approaching the area.\\
	


\needspace{1.5cm}
\label{rt:Pumpkin Spice}
\colorbox{Goldenrod!50}{
\parbox{0.95\linewidth}{
\textbf{
1 Pumpkin Spice* V6/7 \ding{72}\ding{72} 
}}}
\begin{adjustwidth}{0.5cm}{}			
Sit start on the left side of the overhang with left hand on a sharp side pull and right hand on the lower of two side pull rails. Trend right along the roof to an easy topout over a sussy landing.
\end{adjustwidth}




	\setbox0=\hbox{\begin{overpic}[width=0.8\linewidth]{./maps/topos/baseball_c.png}
	\end{overpic}}
	\needspace{\ht0}
	\begin{center}
	\begin{overpic}[width=0.9\linewidth]{./maps/topos/baseball_c.png}\label{tp:Baseball}
	\end{overpic}
	\end{center}


\needspace{1.5cm}
\subsection*{Baseball}\label{bf:Baseball}
This is one of the few boulders that isn't covered in poison oak, but there is quite a lot of it sounding it. Approach with caution.\\
	


\needspace{1.5cm}
\label{rt:Baseball}
\colorbox{green!20}{
\parbox{0.95\linewidth}{
\textbf{
2 Baseball V3- \ding{72} 
}}}
\begin{adjustwidth}{0.5cm}{}			
Sit start with a high left hand on a good dish around the blunt corner and a low right hand pinch. Pull a powerful move to good edges and continue straight up.
\end{adjustwidth}




\needspace{1.5cm}
\label{rt:Bunt}
\colorbox{green!20}{
\parbox{0.95\linewidth}{
\textbf{
3 Bunt V1 \ding{72} 
}}}
\begin{adjustwidth}{0.5cm}{}			
Sit start with both hands in a low bubbly pod. Climb straight up.
\end{adjustwidth}





\newpage
	\end{multicols}
	\setbox0=\hbox{\begin{overpic}[width=0.8\linewidth]{./maps/area/out/bread_c.png}
	\end{overpic}}
	\needspace{\ht0}
	\begin{center}
	\begin{overpic}[width=0.9\linewidth]{./maps/area/out/bread_c.png}\label{sm:Bread loaf/Scratch and Spliff area map}
	\end{overpic}
	\end{center}
	\raggedcolumns
	\begin{multicols}{2}


\section{B - The Bread Loaves/Scratch and Spliff}\label{sa:The Bread Loaves/Scratch and Spliff}
These two boulders are the area's main attraction. Historically some groups have called both boulders Scratch and Spliff while others called them both the Bread Loaves. The modern compromise seems to be that the upper boulder is Scratch and Spliff while the lower boulder is the Bread Loaf.\\




\needspace{1.5cm}
\subsection*{Bread Loaf}\label{bf:Bread Loaf}
	


\needspace{1.5cm}
\label{rt:Bread Loaf Left}
\colorbox{RoyalBlue!20}{
\parbox{0.95\linewidth}{
\textbf{
1 Bread Loaf Left V4 \ding{72}\ding{72} 
}}}
\begin{adjustwidth}{0.5cm}{}			
Stand start on two horizontal edges. Navigate your way to some good lumpy jugs midway up the route and either mantle or side pull your way to the top. Also called Buddha's Belly.
\end{adjustwidth}




\needspace{1.5cm}
\label{rt:Breadwinner}
\colorbox{red!20}{
\parbox{0.95\linewidth}{
\textbf{
2 Breadwinner V10-*  
}}}
\begin{adjustwidth}{0.5cm}{}			
Start as for Bread Loaf Traverse, climb straight up..
\end{adjustwidth}




\needspace{1.5cm}
\label{rt:Bread Loaf Traverse}
\colorbox{RoyalBlue!20}{
\parbox{0.95\linewidth}{
\textbf{
3 Bread Loaf Traverse V5 \ding{72}\ding{72} 
}}}
\begin{adjustwidth}{0.5cm}{}			
stand start with hands matched in the left of two good pods in the lowest diagonal crack. Follow the crack system right with the help of a good hold under the roof. top along the arête. Dabby.
\end{adjustwidth}


\begin{adjustwidth}{0.5cm}{}				
\needspace{3cm}
\textbf{Variations:} \newline

\needspace{1.5cm}
\label{vr:Baker's Dozen}
\colorbox{Goldenrod!50}{
\parbox{0.95\linewidth}{
\textbf{
3a Baker's Dozen V8*  
}}}
\begin{adjustwidth}{0.5cm}{}			
Start as for Bread Loaf Left, traverse into the bread loaf traverse.
\end{adjustwidth}



\end{adjustwidth}

	\end{multicols}
	\setbox0=\hbox{\begin{overpic}[width=0.8\linewidth]{./maps/topos/breadLoaf_c.png}
	\end{overpic}}
	\needspace{\ht0}
	\begin{center}
	\begin{overpic}[width=0.9\linewidth]{./maps/topos/breadLoaf_c.png}\label{tp:Bread Loaf}
	\end{overpic}
	\end{center}
	\raggedcolumns
	\begin{multicols}{2}


\needspace{1.5cm}
\label{rt:Worf}
\colorbox{RoyalBlue!20}{
\parbox{0.95\linewidth}{
\textbf{
4 Worf V5 \ding{72}\ding{72} 
}}}
\begin{adjustwidth}{0.5cm}{}			
Starting from two horizontal cracks a bizarre sequence leads you first left then right as you climb the rounded corner. Some but not all of the difficulty comes from the dab potential.
\end{adjustwidth}


	\setbox0=\hbox{\begin{overpic}[width=0.8\linewidth]{./maps/topos/breadLoaf2_c.png}
	\end{overpic}}
	\needspace{\ht0}
	\begin{center}
	\begin{overpic}[width=0.9\linewidth]{./maps/topos/breadLoaf2_c.png}\label{tp:Bread Loaf2}
	\end{overpic}
	\end{center}



	\end{multicols}
	\setbox0=\hbox{\begin{overpic}[width=0.8\linewidth]{./maps/topos/scratchSpliff2_c.png}
	\end{overpic}}
	\needspace{\ht0}
	\begin{center}
	\begin{overpic}[width=0.9\linewidth]{./maps/topos/scratchSpliff2_c.png}\label{tp:Scratch and Spliff 2}
	\end{overpic}
	\end{center}
	\raggedcolumns
	\begin{multicols}{2}


\needspace{1.5cm}
\subsection*{Scratch and Spliff}\label{bf:Scratch and Spliff}
	


\needspace{1.5cm}
\label{rt:Scratch and Spliff Traverse}
\colorbox{green!20}{
\parbox{0.95\linewidth}{
\textbf{
5 Scratch and Spliff Traverse V3 \ding{72}\ding{72}\ding{72} 
}}}
\begin{adjustwidth}{0.5cm}{}			
Start at the far right of the major horizontal crack (as for Roach) and traverse all the way left topping out along a juggy vertical crack system.
\end{adjustwidth}


\begin{adjustwidth}{0.5cm}{}				
\needspace{3cm}
\textbf{Variations:} \newline

\needspace{1.5cm}
\label{vr:Late Start}
\colorbox{green!20}{
\parbox{0.95\linewidth}{
\textbf{
5a Late Start* V2 \ding{72}\ding{72} 
}}}
\begin{adjustwidth}{0.5cm}{}			
Sit start with juggy holds at the top of a low ramp. Climb straight up into the top of Scratch and Spliff Traverse.
\end{adjustwidth}



\end{adjustwidth}


\needspace{1.5cm}
\label{rt:Scratch}
\colorbox{RoyalBlue!20}{
\parbox{0.95\linewidth}{
\textbf{
6 Scratch V4 \ding{72}\ding{72} 
}}}
\begin{adjustwidth}{0.5cm}{}			
Stand start with right hand on a good hold in the horizontal crack and left hand wrapping around a juggy corner. Jump to a bubbly rail and tick tack your way to the top. Originally this route started as for Scratch and Spliff Traverse.
\end{adjustwidth}



	\setbox0=\hbox{\begin{overpic}[width=0.8\linewidth]{./maps/topos/scratchSpliff_c.png}
	\end{overpic}}
	\needspace{\ht0}
	\begin{center}
	\begin{overpic}[width=0.9\linewidth]{./maps/topos/scratchSpliff_c.png}\label{tp:Scratch and Spliff}
	\end{overpic}
	\end{center}


\needspace{1.5cm}
\label{rt:Spliff}
\colorbox{green!20}{
\parbox{0.95\linewidth}{
\textbf{
7 Spliff V3 \ding{72}\ding{72}\ding{72} \warn
}}}
\begin{adjustwidth}{0.5cm}{}			
Start on a large hanging flake. Climb straight up. Sit start seems possible but wouldn't add much to the experience.
\end{adjustwidth}




\needspace{1.5cm}
\label{rt:Roach}
\colorbox{green!20}{
\parbox{0.95\linewidth}{
\textbf{
8 Roach V0 \ding{72}\ding{72} 
}}}
\begin{adjustwidth}{0.5cm}{}			
Stand start with a good edge in the horizantal crack..
\end{adjustwidth}




\needspace{1.5cm}
\label{rt:For What it's Worth}
\colorbox{green!20}{
\parbox{0.95\linewidth}{
\textbf{
9 For What it's Worth* V2 \ding{72}\ding{72} 
}}}
\begin{adjustwidth}{0.5cm}{}			
Squat start on a low ramp on the NW corner of the boulder using a left hand low on the arête and a right hand side pull. Bump up the arête then dyno to the lip. Dab potential creates a lot of the difficulty.
\end{adjustwidth}



	\setbox0=\hbox{\begin{overpic}[width=0.8\linewidth]{./maps/topos/scratchSpliff3_c.png}
	\end{overpic}}
	\needspace{\ht0}
	\begin{center}
	\begin{overpic}[width=0.9\linewidth]{./maps/topos/scratchSpliff3_c.png}\label{tp:Scratch and Spliff 3}
	\end{overpic}
	\end{center}


\needspace{1.5cm}
\label{rt:Caliban's War}
\colorbox{RoyalBlue!20}{
\parbox{0.95\linewidth}{
\textbf{
10 Caliban's War V6*  
}}}
\begin{adjustwidth}{0.5cm}{}			
Stand start with hand holds in a horizontal crack. Crank one move to the lip. Wouldn't be surprised if this has never been done.
\end{adjustwidth}




\needspace{1.5cm}
\label{rt:Stoned Age}
\colorbox{green!20}{
\parbox{0.95\linewidth}{
\textbf{
11 Stoned Age V2*  
}}}
\begin{adjustwidth}{0.5cm}{}			
It looks like you could easily climb from the horizontal crack to a diagonal crack on the upper right, but the landing is very poor. Older guidebooks indicate that this has been done.
\end{adjustwidth}





\newpage

\section{C - Dr. Strangelove}\label{sa:Dr. Strangelove}
More boulders lay under the canopy beyond the tallus NE of the scratch and spliff area. Although there is a lot of poison oak in the way there is one passage which avoids most of it. From the scratch and spliff boulder walk across jumbled tallus towards the cliff band for ~100' until you pass a large fir tree. From here the distinctive prow of the Dr. Stanglove boulder should be visible throught the trees. Walk more or less directly towards it bushwhacking along a feint trail once you get into the trees. There is much less poison oak under the canopy but it can still be found in some patches.\\




\needspace{1.5cm}
\subsection*{Dr. Strangelove}\label{bf:Dr. Strangelove}
	


\needspace{1.5cm}
\label{rt:Dr. Strangelove}
\colorbox{RoyalBlue!20}{
\parbox{0.95\linewidth}{
\textbf{
1 Dr. Strangelove V6*  
}}}
\begin{adjustwidth}{0.5cm}{}			
(PLACEHOLDER) Also known as "The Hook"
  (No Topo)
\end{adjustwidth}




\needspace{1.5cm}
\label{rt:War Room}
\colorbox{Goldenrod!50}{
\parbox{0.95\linewidth}{
\textbf{
2 War Room V9*  
}}}
\begin{adjustwidth}{0.5cm}{}			
(PLACEHOLDER) Also known as "Andrew's Line" (no, some other Andrew).
  (No Topo)
\end{adjustwidth}





\needspace{1.5cm}
\subsection*{Kick It}\label{bf:Kick It}
	


\needspace{1.5cm}
\label{rt:Kick It}
\colorbox{green!20}{
\parbox{0.95\linewidth}{
\textbf{
3 Kick It V2 \ding{72}\ding{72} 
}}}
\begin{adjustwidth}{0.5cm}{}			
Start standing with left hand on a small edge or on the left arete and right hand undercling a big slopey rib. Climb the clean face using both aretes. Worth doing if you are making the trek out to strange love. Also known as Dishing.
  (No Topo)
\end{adjustwidth}





\newpage

\section{D - Machete Monkey}\label{sa:Machete Monkey}
About 100' east of Dr. Strangelove there is a narrow wash of boulders. Getting here requires a lot of bushwhacking but a faint trail can be followed from Dr. Strangelove to the cliff then back down towards the Machete Monkey boulder. Even without carrying pads navigating this trail is difficult.\\




\needspace{1.5cm}
\subsection*{Machete Monkey}\label{bf:Machete Monkey}
	


\needspace{1.5cm}
\label{rt:Machete Monkey}
\colorbox{green!20}{
\parbox{0.95\linewidth}{
\textbf{
1 Machete Monkey V3*  
}}}
\begin{adjustwidth}{0.5cm}{}			
PLACEHOLDER
  (No Topo)
\end{adjustwidth}


\begin{adjustwidth}{0.5cm}{}				
\needspace{3cm}
\textbf{Variations:} \newline

\needspace{1.5cm}
\label{vr:Machete Man}
\colorbox{RoyalBlue!20}{
\parbox{0.95\linewidth}{
\textbf{
1a Machete Man V5*  
}}}
\begin{adjustwidth}{0.5cm}{}			
PLACEHOLDER
  (No Topo)
\end{adjustwidth}



\end{adjustwidth}



\needspace{1.5cm}
\subsection*{June 24th}\label{bf:June 24th}
	


\needspace{1.5cm}
\label{rt:June 24th}
\colorbox{Goldenrod!50}{
\parbox{0.95\linewidth}{
\textbf{
2 June 24th V7*  
}}}
\begin{adjustwidth}{0.5cm}{}			
PLACEHOLDER
  (No Topo)
\end{adjustwidth}





\needspace{1.5cm}
\subsection*{Young Jui¢y}\label{bf:Young Jui¢y}
	


\needspace{1.5cm}
\label{rt:Young Jui¢y}
\colorbox{Goldenrod!50}{
\parbox{0.95\linewidth}{
\textbf{
3 Young Jui¢y V8*  
}}}
\begin{adjustwidth}{0.5cm}{}			
PLACEHOLDER
  (No Topo)
\end{adjustwidth}





\newpage

\section{E - Middle Garden}\label{sa:Middle Garden}
There are many boulders under the cliffline east of the Machete Monkey area. While there is little to no poison oak in this area the forest has completely overgrown any trails that once existed. Rediscovering and restablishing this sector is a project for the future. Note that approaching this sector via bushwhacking from the road just before the main area parking pullout may be easier than approaching from Machete Monkey.\\





\end{multicols}
\clearpage