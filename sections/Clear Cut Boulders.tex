

\thispagestyle{empty}
\colorlet{shadecolor}{\chapterColor}
\chapter{Clear Cut Boulders}

\fancyhead{}
\lhead[\textcolor{\chapterColor}{\rule[-2pt]{\textwidth}{15pt}}]{\textcolor{\chapterColor}{\rule[-2pt]{\textwidth}{15pt}}\hspace{-\textwidth}\color{white}\hspace{4pt}\protect\thepage\hspace{1ex}-\hspace{1ex}Clear Cut Boulders}
\rhead[\textcolor{\chapterColor}{\rule[-2pt]{\textwidth}{15pt}}\hspace{-\textwidth}\color{white}Clear Cut Boulders \protect\thepage \hspace{4pt}]{\textcolor{\chapterColor}{\rule[-2pt]{\textwidth}{15pt}}}
\fancyhead[RO]{}
\fancyhead[RE]{\color{white}Clear Cut Boulders\hspace{1ex}-\hspace{1ex}\protect\thepage \hspace{4pt}}


\raggedcolumns
\begin{multicols}{2}
\setbox0=\hbox{\includegraphics[width=0.45\linewidth]{./maps/qr//Clear Cut Boulders_qr.png}}% Store image in \box0
\needspace{\ht0}% Need at least the height of \box0
\begin{center}
\includegraphics[width=0.45\linewidth]{./maps/qr//Clear Cut Boulders_qr.png}
\end{center}
\begin{center}
\underline{\textcolor{blue}{\href{http://maps.google.com/maps?q=44.43480966863281,-122.5974049928017}{Navigate to this area}}}
\end{center}


\includegraphics[width=\linewidth]{./maps/plots//Clear Cut Boulders.png}
\end{multicols}
\begin{multicols}{2}

A small patch of boulders has been recently developed on Quartzville about mile closer to town from the main Garden area. Although the rock quality here is not as solid as other areas, there are a few great lines available in this sector. At present (2025) the area has been clear cut and replanted, until the trees grow this area offers all day sun. The fast drying boulders are a great option for winter weather windows.

After parking wherever you can manage on the side of the road search for a faint trail through the buffer of vegetation leading into the Clear Cut.\\



\needspace{6em}





\phantomsection\label{tp:Star Gazer}
	\setbox0=\hbox{\begin{overpic}[width=0.8\linewidth]{./maps/topos/StarGazer_c.png}
	\end{overpic}}
	\needspace{\ht0}
	\begin{center}
	\begin{overpic}[width=0.9\linewidth]{./maps/topos/StarGazer_c.png}
	\end{overpic}
	\end{center}


\needspace{10em}
\subsection*{Star Gazer}\phantomsection\label{bf:Star Gazer}

A small boulder with a large shelf mid way up its face sits at the bottom of the hill.\\



\needspace{2em}
\phantomsection\label{rt:Huygens}
\colorbox{green!20}{
\parbox{0.95\linewidth}{
\hspace{-1ex}\textbf{$\Box$
1 Huygens V2 \ding{73} 
}}}
\begin{adjustwidth}{1.3em}{}			

Climb the left arete of the boulder through the big ledge. This route used to have a more coherent start, but a key foothold has broken and establishing is now very difficult.
\end{adjustwidth}




\needspace{2em}
\phantomsection\label{rt:Fleming}
\colorbox{green!20}{
\parbox{0.95\linewidth}{
\hspace{-1ex}\textbf{$\Box$
2 Fleming V3 \ding{72} 
}}}
\begin{adjustwidth}{1.3em}{}			

Sit start with holds low on the center of the main face of the boulder.
\end{adjustwidth}




\needspace{2em}
\phantomsection\label{rt:Galileo}
\colorbox{green!20}{
\parbox{0.95\linewidth}{
\hspace{-1ex}\textbf{$\Box$
3 Galileo V3 \ding{72} 
}}}
\begin{adjustwidth}{1.3em}{}			

Sit start on the excavated right side of the boulder using a low left and undercling and a right hand sidepull. Starting with slightly higher hand holds may result in a more enjoyable experience.
\end{adjustwidth}




	\end{multicols}
\phantomsection\label{tp:Dig Dug}
	\setbox0=\hbox{\begin{overpic}[width=0.8\linewidth]{./maps/topos/DigDug_c.png}
	\end{overpic}}
	\needspace{\ht0}
	\begin{center}
	\begin{overpic}[width=0.9\linewidth]{./maps/topos/DigDug_c.png}
	\end{overpic}
	\end{center}

	\begin{multicols}{2}

\needspace{10em}
\subsection*{Dig Dug}\phantomsection\label{bf:Dig Dug}

This boulder features a well excavated down hill face and a little stone hat.\\



\needspace{2em}
\phantomsection\label{rt:Skrunch}
\colorbox{green!20}{
\parbox{0.95\linewidth}{
\hspace{-1ex}\textbf{$\Box$
4 Skrunch V3 \ding{72} 
}}}
\begin{adjustwidth}{1.3em}{}			

Squat start on the obvious low ledge and skrunch yourself straight up.
\end{adjustwidth}




\needspace{2em}
\phantomsection\label{rt:Mud Flaps}
\colorbox{RoyalBlue!20}{
\parbox{0.95\linewidth}{
\hspace{-1ex}\textbf{$\Box$
5 Mud Flaps V5 \ding{72}\ding{72} 
}}}
\begin{adjustwidth}{1.3em}{}			

Awkward sit start on the big sloper lump. Some committing moves through small crimps lead up and left.
\end{adjustwidth}


\begin{adjustwidth}{0.5cm}{}				
\needspace{4em}
\textbf{Variations:} \newline

\needspace{2em}
\phantomsection\label{vr:Mud Puppy}
\colorbox{RoyalBlue!20}{
\parbox{0.95\linewidth}{
\hspace{-1ex}\textbf{$\Box$
5a Mud Puppy V4 \ding{72}\ding{72} 
}}}
\begin{adjustwidth}{1.3em}{}			

Skip the first few moves of Mud Flaps and start by squatting onto small crimps instead.
  (No Topo)
\end{adjustwidth}



\end{adjustwidth}


\needspace{2em}
\phantomsection\label{rt:Fly Guy}
\colorbox{green!20}{
\parbox{0.95\linewidth}{
\hspace{-1ex}\textbf{$\Box$
6 Fly Guy V2 \ding{72}\ding{72} 
}}}
\begin{adjustwidth}{1.3em}{}			

Stand start on a head height pair of edges in the center of the face. A few gymnastic moves bring you to the top.
\end{adjustwidth}


\begin{adjustwidth}{0.5cm}{}				
\needspace{4em}
\textbf{Variations:} \newline

\needspace{2em}
\phantomsection\label{vr:Fly Guy SDS}
\colorbox{RoyalBlue!20}{
\parbox{0.95\linewidth}{
\hspace{-1ex}\textbf{$\Box$
6a Fly Guy SDS V4 \ding{72} 
}}}
\begin{adjustwidth}{1.3em}{}			

Sit start on Mud Flaps and link into Fly Guy.
\end{adjustwidth}



\end{adjustwidth}


\needspace{2em}
\phantomsection\label{rt:Tear Duct}
\colorbox{RoyalBlue!20}{
\parbox{0.95\linewidth}{
\hspace{-1ex}\textbf{$\Box$
7 Tear Duct V4 \ding{72}\ding{72} 
}}}
\begin{adjustwidth}{1.3em}{}			

Climb the slopey rail from a good flat crimp.
\end{adjustwidth}


\begin{adjustwidth}{0.5cm}{}				
\needspace{4em}
\textbf{Variations:} \newline

\needspace{2em}
\phantomsection\label{vr:Tear Duct Direct}
\colorbox{green!20}{
\parbox{0.95\linewidth}{
\hspace{-1ex}\textbf{$\Box$
7a Tear Duct Direct V1 \ding{72} 
}}}
\begin{adjustwidth}{1.3em}{}			

Start as for Tear Duct but climb straight up.
\end{adjustwidth}



\end{adjustwidth}


\needspace{2em}
\phantomsection\label{rt:Tear Duct Finish}
\colorbox{green!20}{
\parbox{0.95\linewidth}{
\hspace{-1ex}\textbf{$\Box$
8 Tear Duct Finish V1  \warn\warn
}}}
\begin{adjustwidth}{1.3em}{}			

Climb the little boulder above Dig Dug either as a standalone route or as a second pitch continuing any of the routes below.
\end{adjustwidth}




\phantomsection\label{tp:Grizzly Face}
	\setbox0=\hbox{\begin{overpic}[width=0.8\linewidth]{./maps/topos/GrizzlyFace_c.png}
	\end{overpic}}
	\needspace{\ht0}
	\begin{center}
	\begin{overpic}[width=0.9\linewidth]{./maps/topos/GrizzlyFace_c.png}
	\end{overpic}
	\end{center}


\needspace{10em}
\subsection*{Grizzly Face}\phantomsection\label{bf:Grizzly Face}

A long low angle boulder next to the Bear Back.\\



\needspace{2em}
\phantomsection\label{rt:Baloo}
\colorbox{green!20}{
\parbox{0.95\linewidth}{
\hspace{-1ex}\textbf{$\Box$
9 Baloo V2  
}}}
\begin{adjustwidth}{1.3em}{}			

Sit start on the left side of the downhill face.
\end{adjustwidth}




\needspace{2em}
\phantomsection\label{rt:Br'er Bear}
\colorbox{green!20}{
\parbox{0.95\linewidth}{
\hspace{-1ex}\textbf{$\Box$
10 Br'er Bear V2  
}}}
\begin{adjustwidth}{1.3em}{}			

Sit start on the right side of the downhill face.
\end{adjustwidth}




	\end{multicols}
\phantomsection\label{tp:Bear Back}
	\setbox0=\hbox{\begin{overpic}[width=0.8\linewidth]{./maps/topos/BearBack_c.png}
	\end{overpic}}
	\needspace{\ht0}
	\begin{center}
	\begin{overpic}[width=0.9\linewidth]{./maps/topos/BearBack_c.png}
	\end{overpic}
	\end{center}

	\begin{multicols}{2}

\needspace{10em}
\subsection*{Bear Back}\phantomsection\label{bf:Bear Back}

Look for this unassuming slab midway up the hill.\\



\needspace{2em}
\phantomsection\label{rt:Bald Bear}
\colorbox{green!20}{
\parbox{0.95\linewidth}{
\hspace{-1ex}\textbf{$\Box$
11 Bald Bear V2 \ding{72}\ding{72} 
}}}
\begin{adjustwidth}{1.3em}{}			

Sit start on a prominent jug ledge.
\end{adjustwidth}




\needspace{2em}
\phantomsection\label{rt:Wally}
\colorbox{green!20}{
\parbox{0.95\linewidth}{
\hspace{-1ex}\textbf{$\Box$
12 Wally V3 \ding{72}\ding{72} \warn
}}}
\begin{adjustwidth}{1.3em}{}			

Stand start on a left facing flake. Trend left along the slopely lip before mantling onto the slab.
\end{adjustwidth}




\needspace{2em}
\phantomsection\label{rt:Roman Craig}
\colorbox{green!20}{
\parbox{0.95\linewidth}{
\hspace{-1ex}\textbf{$\Box$
13 Roman Craig V1 \ding{72}\ding{72} \warn
}}}
\begin{adjustwidth}{1.3em}{}			

Stand start as for Wally and climb straight up.
\end{adjustwidth}




\needspace{2em}
\phantomsection\label{rt:Chet Ripley}
\colorbox{green!20}{
\parbox{0.95\linewidth}{
\hspace{-1ex}\textbf{$\Box$
14 Chet Ripley V0 \ding{72}\ding{72} \warn
}}}
\begin{adjustwidth}{1.3em}{}			

Climb straight up the far right side of the boulder
\end{adjustwidth}




\phantomsection\label{tp:Double Stump}
	\setbox0=\hbox{\begin{overpic}[width=0.8\linewidth]{./maps/topos/DoubleStump_c.png}
	\end{overpic}}
	\needspace{\ht0}
	\begin{center}
	\begin{overpic}[width=0.9\linewidth]{./maps/topos/DoubleStump_c.png}
	\end{overpic}
	\end{center}


\needspace{10em}
\subsection*{Double Stump}\phantomsection\label{bf:Double Stump}

A large boulder with a prominent down hill face can be recognized by the titular double stumps which adorn it (one above it and on below).\\



\needspace{2em}
\phantomsection\label{rt:Ballistics Dummy}
\colorbox{RoyalBlue!20}{
\parbox{0.95\linewidth}{
\hspace{-1ex}\textbf{$\Box$
15 Ballistics Dummy V4 \ding{72}\ding{72} 
}}}
\begin{adjustwidth}{1.3em}{}			

Squat start on a large left facing rail. Climb the left side of the boulder using off balance holds.
\end{adjustwidth}




\needspace{2em}
\phantomsection\label{rt:Double Stump Project}
\colorbox{black!20}{
\parbox{0.95\linewidth}{
\hspace{-1ex}\textbf{$\Box$
16 Double Stump Project V?  
}}}
\begin{adjustwidth}{1.3em}{}			

A sit start in a small cave on the right side of the boulder leads to an obvious line of crimps up the center of the face.
\end{adjustwidth}


\begin{adjustwidth}{0.5cm}{}				
\needspace{4em}
\textbf{Variations:} \newline

\needspace{2em}
\phantomsection\label{vr:Double Stump Project Right}
\colorbox{black!20}{
\parbox{0.95\linewidth}{
\hspace{-1ex}\textbf{$\Box$
16a Double Stump Project Right V?  
}}}
\begin{adjustwidth}{1.3em}{}			

Start as for the Double stump project but climb straight up the blunt arete over a bad landing.
\end{adjustwidth}



\end{adjustwidth}

\phantomsection\label{tp:Double Stump 2}
	\setbox0=\hbox{\begin{overpic}[width=0.8\linewidth]{./maps/topos/DoubleStump2_c.png}
	\end{overpic}}
	\needspace{\ht0}
	\begin{center}
	\begin{overpic}[width=0.9\linewidth]{./maps/topos/DoubleStump2_c.png}
	\end{overpic}
	\end{center}


\needspace{2em}
\phantomsection\label{rt:Chump Mantle}
\colorbox{green!20}{
\parbox{0.95\linewidth}{
\hspace{-1ex}\textbf{$\Box$
17 Chump Mantle V1 \ding{72}\ding{72}\ding{72} 
}}}
\begin{adjustwidth}{1.3em}{}			

Sit start on a blocky right facing rail. Throw to the obvious flake and perform a technical mantle.
\end{adjustwidth}




	\end{multicols}
\phantomsection\label{tp:Cuneiform}
	\setbox0=\hbox{\begin{overpic}[width=0.8\linewidth]{./maps/topos/Cuneiform_c.png}
	\end{overpic}}
	\needspace{\ht0}
	\begin{center}
	\begin{overpic}[width=0.9\linewidth]{./maps/topos/Cuneiform_c.png}
	\end{overpic}
	\end{center}

	\begin{multicols}{2}

\needspace{10em}
\subsection*{Cuneiform}\phantomsection\label{bf:Cuneiform}

A wedge shaped boulder sits at the top of the hill.\\



\needspace{2em}
\phantomsection\label{rt:Cuneiform1}
\colorbox{green!20}{
\parbox{0.95\linewidth}{
\hspace{-1ex}\textbf{$\Box$
18 Unknown V0  
}}}
\begin{adjustwidth}{1.3em}{}			

The ledgy left face of the boulder looks pretty trivial to climb.
\end{adjustwidth}




\needspace{2em}
\phantomsection\label{rt:Walk the Line}
\colorbox{green!20}{
\parbox{0.95\linewidth}{
\hspace{-1ex}\textbf{$\Box$
19 Walk the Line V2 \ding{72}\ding{72} \warn
}}}
\begin{adjustwidth}{1.3em}{}			

Climb the tallest part of the boulder from a stand start. If this route cleans up it will be a classic.
\end{adjustwidth}




\needspace{2em}
\phantomsection\label{rt:Cuneiform Project}
\colorbox{black!20}{
\parbox{0.95\linewidth}{
\hspace{-1ex}\textbf{$\Box$
20 Cuneiform Project V?  
}}}
\begin{adjustwidth}{1.3em}{}			

An obvious line follows the series of rails along the rise of the boulders wedge shaped face.
\end{adjustwidth}




\needspace{2em}
\phantomsection\label{rt:Eridu}
\colorbox{green!20}{
\parbox{0.95\linewidth}{
\hspace{-1ex}\textbf{$\Box$
21 Eridu V1 \ding{72} 
}}}
\begin{adjustwidth}{1.3em}{}			

climb the right hand side of the boulder to and early exit.
\end{adjustwidth}






\end{multicols}
\clearpage